  \documentclass[journal,12pt,twocolumn]{IEEEtran}
%
\usepackage{setspace}
\usepackage{textcomp}
\usepackage{gensymb}
%\doublespacing
\singlespacing

\usepackage[cmex10]{amsmath}
\usepackage{amsthm}
%\usepackage{iithtlc}
\usepackage{mathrsfs}
\usepackage{txfonts}
\usepackage{stfloats}
\usepackage{bm}
\usepackage{cite}
\usepackage{cases}
\usepackage{subfig}
%\usepackage{xtab}
\usepackage{longtable}
\usepackage{multirow}
%\usepackage{algorithm}
%\usepackage{algpseudocode}
\usepackage{enumitem}
\usepackage{mathtools}
\usepackage{steinmetz}
\usepackage{tikz}
\usepackage{circuitikz}
\usepackage{verbatim}
\usepackage{tfrupee}
\usepackage[breaklinks=true]{hyperref}
%\usepackage{stmaryrd}
\usepackage{tkz-euclide} % loads  TikZ and tkz-base
%\usetkzobj{all}
\usetikzlibrary{calc,math}
\usepackage{listings}
    \usepackage{color}                                            %%
    \usepackage{array}                                            %%
    \usepackage{longtable}                                        %%
    \usepackage{calc}                                             %%
    \usepackage{multirow}                                         %%
    \usepackage{hhline}                                           %%
    \usepackage{ifthen}                                           %%
  %optionally (for landscape tables embedded in another document): %%
    \usepackage{lscape}     
\usepackage{multicol}
\usepackage{chngcntr}
%\usepackage{enumerate}

%\usepackage{wasysym}
%\newcounter{MYtempeqncnt}
\DeclareMathOperator*{\Res}{Res}
%\renewcommand{\baselinestretch}{2}
\renewcommand\thesection{\arabic{section}}
\renewcommand\thesubsection{\thesection.\arabic{subsection}}
\renewcommand\thesubsubsection{\thesubsection.\arabic{subsubsection}}

\renewcommand\thesectiondis{\arabic{section}}
\renewcommand\thesubsectiondis{\thesectiondis.\arabic{subsection}}
\renewcommand\thesubsubsectiondis{\thesubsectiondis.\arabic{subsubsection}}

% correct bad hyphenation here
\hyphenation{op-tical net-works semi-conduc-tor}
\def\inputGnumericTable{}                                 %%

\lstset{
%language=C,
frame=single, 
breaklines=true,
columns=fullflexible
}
\newenvironment{amatrix}[1]{%
  \left(\begin{array}{@{}*{#1}{c}|c@{}}
}{%
  \end{array}\right)
}
\DeclarePairedDelimiter\abs{\lvert}{\rvert}%
\DeclarePairedDelimiter\norm{\lVert}{\rVert}%

% Swap the definition of \abs* and \norm*, so that \abs
% and \norm resizes the size of the brackets, and the 
% starred version does not.
\makeatletter
\let\oldabs\abs
\def\abs{\@ifstar{\oldabs}{\oldabs*}}
%
\let\oldnorm\norm
\def\norm{\@ifstar{\oldnorm}{\oldnorm*}}
\makeatother

\newtheorem{theorem}{Theorem}[section]
\newtheorem{problem}{Problem}
\newtheorem{proposition}{Proposition}[section]
\newtheorem{lemma}{Lemma}[section]
\newtheorem{corollary}[theorem]{Corollary}
\newtheorem{example}{Example}[section]
\newtheorem{definition}[problem]{Definition}
%\newtheorem{thm}{Theorem}[section] 
%\newtheorem{defn}[thm]{Definition}
%\newtheorem{algorithm}{Algorithm}[section]
%\newtheorem{cor}{Corollary}
\newcommand{\BEQA}{\begin{eqnarray}}
\newcommand{\EEQA}{\end{eqnarray}}
\newcommand{\define}{\stackrel{\triangle}{=}}
\bibliographystyle{IEEEtran}
%\bibliographystyle{ieeetr}
\providecommand{\mbf}{\mathbf}
\providecommand{\pr}[1]{\ensuremath{\Pr\left(#1\right)}}
\providecommand{\qfunc}[1]{\ensuremath{Q\left(#1\right)}}
\providecommand{\sbrak}[1]{\ensuremath{{}\left[#1\right]}}
\providecommand{\lsbrak}[1]{\ensuremath{{}\left[#1\right.}}
\providecommand{\rsbrak}[1]{\ensuremath{{}\left.#1\right]}}
\providecommand{\brak}[1]{\ensuremath{\left(#1\right)}}
\providecommand{\lbrak}[1]{\ensuremath{\left(#1\right.}}
\providecommand{\rbrak}[1]{\ensuremath{\left.#1\right)}}
\providecommand{\cbrak}[1]{\ensuremath{\left\{#1\right\}}}
\providecommand{\lcbrak}[1]{\ensuremath{\left\{#1\right.}}
\providecommand{\rcbrak}[1]{\ensuremath{\left.#1\right\}}}

\providecommand{\system}{\overset{\mathcal{H}}{ \longleftrightarrow}}
	%\newcommand{\solution}[2]{\textbf{Solution:}{#1}}
\newcommand{\solution}{\noindent \textbf{Solution: }}
\newcommand{\cosec}{\,\text{cosec}\,}
\providecommand{\dec}[2]{\ensuremath{\overset{#1}{\underset{#2}{\gtrless}}}}
\newcommand{\myvec}[1]{\ensuremath{\begin{pmatrix}#1\end{pmatrix}}}
\newcommand{\mydet}[1]{\ensuremath{\begin{vmatrix}#1\end{vmatrix}}}
%\numberwithin{equation}{section}
\numberwithin{equation}{subsection}
%\numberwithin{problem}{section}
%\numberwithin{definition}{section}
\makeatletter
\@addtoreset{figure}{problem}
\makeatother
\let\StandardTheFigure\thefigure
\let\vec\mathbf
\usepackage{mathtools, nccmath}

\begin{document}

\begin{center}
\huge Assignment 7\\

\large Shaik Zeeshan Ali\\
\large AI20MTECH11001\\
\end{center}
\begin{abstract}
This document is about finding the Qr decomposition of a matrix and finding the solution using singular value decomposition.
\end{abstract}
Download all python codes from 
\begin{lstlisting}
https://github.com/Zeeshan-IITH/IITH-EE5609/new/master/codes
\end{lstlisting}

and latex-tikz codes from 
\begin{lstlisting}
https://github.com/Zeeshan-IITH/IITH-EE5609
\end{lstlisting}
\section{problem}
Given the equation of a parabola is
\begin{align}
    \vec{x}^T\myvec{4&-2\\-2&1}\vec{x}+2\myvec{-6&3}\vec{x}+9=0
\end{align}
Find the QR-decomposition of the matrix $\myvec{4&-2\\-2&1}$.
Find the vertex of the parabola using singular value decomposition.
\section{construction}
The vertex of the parabola can be found by using
\begin{align}
    \myvec{\vec{u}^T+\eta p_1^T\\\vec{V}}c=\myvec{-f\\\eta p_1-\vec{u}}
\end{align}
where
\begin{align}
    \eta=p_1^T\vec{u}
\end{align}
\section{QR-decomposition}
The QR decomposition of the matrix will be done by using Gram-schmidt process of finding an orthogonal basis for the column space of the matrix.The matrix $\vec{V}$ can be written as a product of two matrices as $\vec{V}=\vec{Q}\vec{R}$ where
\begin{align}
    \vec{Q}=\myvec{e_1&e_2}\\
    \vec{R}=\myvec{r_1&r_2\\0&r_3}
\end{align}
using gram-schmidt process,let $\vec{V}=\myvec{a_1&a_2}$, we get
\begin{align}
    u_1&=a_1=\myvec{4\\-2}\\
    e_1&=\frac{u_1}{\norm{u_1}}=\frac{1}{\sqrt{20}}\myvec{4\\-2}\\
    u_2&=a_2-\frac{u_1^Ta_2}{\norm{u_1}^2}u_1\\
    &=\myvec{-2\\1}+\frac{1}{2}\myvec{4\\-2}\\
    &=0\\
    e_2&=0
\end{align}
The elements of the matrix $\vec{R}$ can be found by taking the projection of $e_1,e_2$ on to $a_1,a_2$.
\begin{align}
    r_1&=e_1^Ta_1=\frac{1}{\sqrt{20}}\myvec{4&-2}\myvec{4\\-2}=\sqrt{20}\\
    r_2&=e_1^Ta_2=\frac{1}{\sqrt{20}}\myvec{4&-2}\myvec{-2\\1}=-\sqrt{5}\\
    r_3&=e_2^Ta_2=0
\end{align}
Therefore the QR decomposition will be
\begin{align}
    \myvec{4&-2\\-2&1}=\myvec{\frac{4}{\sqrt{20}}&0\\-\frac{2}{\sqrt{20}}&0}\myvec{\sqrt{20}&-\sqrt{5}\\0&0}
\end{align}
Since the column vector $e_2=0$,we can write the QR decomposition as
\begin{align}
    \myvec{4&-2\\-2&1}=\myvec{\frac{4}{\sqrt{20}}\\-\frac{2}{\sqrt{20}}}\myvec{\sqrt{20}&-\sqrt{5}}
\end{align}
\section{singular value decomposition}
The characteristic equation of the matrix $\vec{V}=\myvec{4&-2\\-2&1}$ is
\begin{align}
    \det \brak{\vec{V}-\lambda\vec{I}}&=\det \myvec{4-\lambda&-2\\-2&1-\lambda}\\
    &=\lambda^2-5\lambda=0\\
    &\lambda_1=0,\lambda_2=5
\end{align}
The eigen vector corresponding to $\lambda_1$ is in the nullspace of $\vec{V}-0\vec{I}$
\begin{align}
    \myvec{4&-2\\-2&1}\xleftrightarrow{R_2=R_2+\frac{R_1}{2}}\myvec{4&-2\\0&0}\\
    p_1=\frac{1}{\sqrt{5}}\myvec{1\\2}
\end{align}
The eigen vector corresponding to $\lambda_2$ is is in the nullspace of $\vec{V}-5\vec{I}$
\begin{align}
    \myvec{-1&-2\\-2&-4}\xleftrightarrow{R_2=R_2-2R_1}\myvec{-1&-2\\0&0}\\
    p_2=\frac{1}{\sqrt{5}}\myvec{2\\-1}
\end{align}
Diagonalising the matrix $\vec{V}$ using $\vec{P}=\myvec{p_1&p_2}$ we get
\begin{align}
    \vec{D}&=\vec{P}^T\vec{V}\vec{P}\\
    &=\frac{1}{5}\myvec{1&2\\2&-1}\myvec{4&-2\\-2&1}\myvec{1&2\\2&-1}\\
    &=\myvec{0&0\\0&5}
\end{align}
The standard equation of the parabola is 
\begin{align}
    \vec{y}^T\vec{D}\vec{y}=-2\eta\myvec{1&0}\vec{y}
\end{align}
Where $\eta$ can be calculated as
\begin{align}
    \eta&=p_1^T\vec{u}\\
    &=\frac{1}{\sqrt{5}}\myvec{1&2}\myvec{-6\\3}\\
    &=0
\end{align}
The vertex of the parabola $\vec{c}$ can be calculated from
\begin{align}
    \myvec{\vec{u}^T+2\eta p_1^T\\\vec{V}}\vec{c}=\myvec{-f\\2\eta p_1-\vec{u}}\\
    \myvec{\vec{u}^T\\\vec{V}}\vec{c}=\myvec{-f\\-\vec{u}}\\
    \myvec{6&-3\\4&-2\\-2&1}\vec{c}=\myvec{9\\-6\\3}\label{eq:4}
\end{align}
This equation \eqref{eq:4} is of the form $\vec{A}\vec{c}=\vec{b}$,this can be calculated by using the singular value decomposition of $\vec{A}$,where
\begin{align}
    \vec{A}=\vec{U}\vec{S}\vec{V}^T
\end{align}
The eigen vectors of $\vec{A}^T\vec{A}$ are columns of $\vec{V}$ and the eigen vectors of $\vec{A}\vec{A}^T$ are columns of $\vec{U}$ and the matrix $\vec{S}$ is a diagonal matrix with entries as the singular values of $\vec{A}^T\vec{A}$.
\begin{align}
    \vec{U}\vec{S}\vec{V}^T\vec{c}=\vec{b}\\
    \vec{c}=\vec{V}\vec{S}_+\vec{U}^T\vec{b}
\end{align}
where $\vec{A}_+=\vec{V}\vec{S}_+\vec{U}^T$ is the moore-penrose pseudo-inverse of $\vec{S}$.
Calculating $\vec{A}\vec{A}^T$,we get
\begin{align}
    &=\myvec{6&-3\\4&-2\\-2&1}\myvec{6&-3\\4&-2\\-2&1}^T\\
    &=\myvec{6&-3\\4&-2\\-2&1}\myvec{6&4&-2\\-3&-2&1}\\
    &=\myvec{45&30&-15\\30&20&-10\\-15&-10&5}
\end{align}
The eigen values and eigen vectors for $\vec{A}\vec{A}^T$ are
\begin{align}
    \det \brak{\vec{A}\vec{A}^T-\lambda\vec{I}}=0\\
    \begin{vmatrix}45-\lambda&30&-15\\30&20-\lambda&-10\\-15&-10&5-\lambda\end{vmatrix}=0\\
    \lambda^3-70\lambda^2=0\\
    \lambda_1=70,\lambda_2=0
\end{align}
The eigen vectors corresponding to the eigen values in the normalized form are
\begin{align}
    \vec{u}_1=\frac{1}{\sqrt{14}}\myvec{-3\\-2\\1}\\
    \vec{u}_2=\frac{3}{\sqrt{13}}\myvec{-\frac{2}{3}\\1\\0}\\
    \vec{u}_3=\frac{3}{\sqrt{10}}\myvec{\frac{1}{3}\\0\\1}
\end{align}
Thus we get
\begin{align}
    \vec{U}=\myvec{-\frac{3}{\sqrt{14}}&-\frac{2}{\sqrt{13}}&\frac{1}{\sqrt{10}}\\-\frac{2}{\sqrt{14}}&\frac{3}{\sqrt{13}}&0\\\frac{1}{\sqrt{14}}&0&\frac{3}{\sqrt{10}}}
\end{align}
The matrix $\vec{S}$ corresponding to the singular values is
\begin{align}
    \vec{S}=\myvec{\sqrt{70}&0\\0&0\\0&0}
\end{align}
Calculating $\vec{A}^T\vec{A}$,we get
\begin{align}
    &=\myvec{6&-3\\4&-2\\-2&1}^T\myvec{6&-3\\4&-2\\-2&1}\\
    &=\myvec{6&4&-2\\-3&-2&1}\myvec{6&-3\\4&-2\\-2&1}\\
    &=\myvec{56&-28\\-28&14}
\end{align}
The eigen values and eigen vectors for $\vec{A}^T\vec{A}$ are
\begin{align}
    \det \brak{\vec{A}^T\vec{A}-\lambda\vec{I}}=0\\
    \begin{vmatrix}56-\lambda&-28\\-28&14-\lambda\end{vmatrix}\\
    \lambda^2-70\lambda=0\\
    \lambda_1=70,\lambda_2=0
\end{align}
The eigen vectors corresponding to the eigen values in the normalized form are
\begin{align}
    \vec{v}_1=\frac{1}{\sqrt{5}}\myvec{-2\\1}\\
    \vec{v}_2=\frac{2}{\sqrt{5}}\myvec{\frac{1}{2}\\1}
\end{align}
The matrix $\vec{V}$ will be
\begin{align}
    \vec{V}=\myvec{-\frac{2}{\sqrt{5}}&\frac{1}{\sqrt{5}}\\\frac{1}{\sqrt{5}}&\frac{2}{\sqrt{5}}}
\end{align}
The moore-penrose pseudo inverse of the matrix $\vec{A}_+=\vec{V}\vec{S}_+\vec{U}^T$,where $\vec{S}_+$ is obtained taking the reciprocal of the non-zero entries of the matrix $\vec{S}$ be the 
\begin{align}
    \vec{S}_+=\myvec{\frac{1}{\sqrt{70}}&0&0\\0&0&0}
\end{align}
The moore-penrose pseudo inverse will be
\begin{align}
    \vec{A}_+=\myvec{-\frac{2}{\sqrt{5}}&\frac{1}{\sqrt{5}}\\\frac{1}{\sqrt{5}}&\frac{2}{\sqrt{5}}}\myvec{\frac{1}{\sqrt{70}}&0&0\\0&0&0}\myvec{-\frac{3}{\sqrt{14}}&-\frac{2}{\sqrt{13}}&\frac{1}{\sqrt{10}}\\-\frac{2}{\sqrt{14}}&\frac{3}{\sqrt{13}}&0\\\frac{1}{\sqrt{14}}&0&\frac{3}{\sqrt{10}}}^T\\
    =\myvec{-\frac{2}{\sqrt{350}}&0&0\\\frac{1}{\sqrt{350}}&0&0}\myvec{-\frac{3}{\sqrt{14}}&-\frac{2}{\sqrt{14}}&\frac{1}{\sqrt{14}}\\-\frac{2}{\sqrt{13}}&\frac{3}{\sqrt{13}}&0\\\frac{1}{\sqrt{10}}&0&\frac{3}{\sqrt{10}}}\\
    =\myvec{\frac{3}{35}&\frac{2}{35}&-\frac{1}{35}\\-\frac{3}{70}&-\frac{2}{70}&\frac{1}{70}}
\end{align}
The value of $\vec{c}$ can now be calculated as
\begin{align}
    \vec{c}=\vec{A}_+\vec{b}\\
    =\vec{V}\vec{S}_+\vec{U}^T\vec{b}\\
    =\myvec{\frac{3}{35}&\frac{2}{35}&-\frac{1}{35}\\-\frac{3}{70}&-\frac{2}{70}&\frac{1}{70}}\myvec{9\\-6\\3}\\
    =\myvec{\frac{12}{35}\\-\frac{6}{35}}
\end{align}
\end{document}

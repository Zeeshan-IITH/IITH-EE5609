\documentclass[journal,12pt]{IEEEtran}
\usepackage{longtable}
\usepackage{setspace}
\usepackage{gensymb}
\singlespacing
\usepackage[cmex10]{amsmath}
\newcommand\myemptypage{
	\null
	\thispagestyle{empty}
	\addtocounter{page}{-1}
	\newpage
}
\usepackage{amsthm}
\usepackage{mdframed}
\usepackage{mathrsfs}
\usepackage{txfonts}
\usepackage{stfloats}
\usepackage{bm}
\usepackage{cite}
\usepackage{cases}
\usepackage{subfig}

\usepackage{longtable}
\usepackage{multirow}


\usepackage{enumitem}
\usepackage{mathtools}
\usepackage{steinmetz}
\usepackage{tikz}
\usepackage{circuitikz}
\usepackage{verbatim}
\usepackage{tfrupee}
\usepackage[breaklinks=true]{hyperref}
\usepackage{graphicx}
\usepackage{tkz-euclide}

\usetikzlibrary{calc,math}
\usepackage{listings}
    \usepackage{color}                                            %%
    \usepackage{array}                                            %%
    \usepackage{longtable}                                        %%
    \usepackage{calc}                                             %%
    \usepackage{multirow}                                         %%
    \usepackage{hhline}                                           %%
    \usepackage{ifthen}                                           %%
    \usepackage{lscape}     
\usepackage{multicol}
\usepackage{chngcntr}

\DeclareMathOperator*{\Res}{Res}

\renewcommand\thesection{\arabic{section}}
\renewcommand\thesubsection{\thesection.\arabic{subsection}}
\renewcommand\thesubsubsection{\thesubsection.\arabic{subsubsection}}

\renewcommand\thesectiondis{\arabic{section}}
\renewcommand\thesubsectiondis{\thesectiondis.\arabic{subsection}}
\renewcommand\thesubsubsectiondis{\thesubsectiondis.\arabic{subsubsection}}


\hyphenation{op-tical net-works semi-conduc-tor}
\def\inputGnumericTable{}                                 %%

\lstset{
%language=C,
frame=single, 
breaklines=true,
columns=fullflexible
}
\begin{document}
\onecolumn

\newtheorem{theorem}{Theorem}[section]
\newtheorem{problem}{Problem}
\newtheorem{proposition}{Proposition}[section]
\newtheorem{lemma}{Lemma}[section]
\newtheorem{corollary}[theorem]{Corollary}
\newtheorem{example}{Example}[section]
\newtheorem{definition}[problem]{Definition}

\newcommand{\BEQA}{\begin{eqnarray}}
\newcommand{\EEQA}{\end{eqnarray}}
\newcommand{\define}{\stackrel{\triangle}{=}}
\bibliographystyle{IEEEtran}
\raggedbottom
\setlength{\parindent}{0pt}
\providecommand{\mbf}{\mathbf}
\providecommand{\pr}[1]{\ensuremath{\Pr\left(#1\right)}}
\providecommand{\qfunc}[1]{\ensuremath{Q\left(#1\right)}}
\providecommand{\sbrak}[1]{\ensuremath{{}\left[#1\right]}}
\providecommand{\lsbrak}[1]{\ensuremath{{}\left[#1\right.}}
\providecommand{\rsbrak}[1]{\ensuremath{{}\left.#1\right]}}
\providecommand{\brak}[1]{\ensuremath{\left(#1\right)}}
\providecommand{\lbrak}[1]{\ensuremath{\left(#1\right.}}
\providecommand{\rbrak}[1]{\ensuremath{\left.#1\right)}}
\providecommand{\cbrak}[1]{\ensuremath{\left\{#1\right\}}}
\providecommand{\lcbrak}[1]{\ensuremath{\left\{#1\right.}}
\providecommand{\rcbrak}[1]{\ensuremath{\left.#1\right\}}}
\theoremstyle{remark}
\renewcommand{\arraystretch}{2}
\newtheorem{rem}{Remark}
\newcommand{\sgn}{\mathop{\mathrm{sgn}}}
\providecommand{\abs}[1]{\mathbf{\left\vert#1\right\vert}}
\providecommand{\res}[1]{\Res\displaylimits_{#1}} 
\providecommand{\norm}[1]{\mathbf{\left\lVert#1\right\rVert}}
%\providecommand{\norm}[1]{\lVert#1\rVert}
\providecommand{\mtx}[1]{\mathbf{#1}}
\providecommand{\mean}[1]{\mathbf{E\left[ #1 \right]}}
\providecommand{\fourier}{\overset{\mathcal{F}}{ \rightleftharpoons}}
%\providecommand{\hilbert}{\overset{\mathcal{H}}{ \rightleftharpoons}}
\providecommand{\system}{\overset{\mathcal{H}}{ \longleftrightarrow}}
	%\newcommand{\solution}[2]{\textbf{Solution:}{#1}}
\newcommand{\solution}{\noindent \textbf{Solution: }}
\newcommand{\cosec}{\,\text{cosec}\,}
\providecommand{\dec}[2]{\ensuremath{\overset{#1}{\underset{#2}{\gtrless}}}}
\newcommand{\myvec}[1]{\ensuremath{\begin{pmatrix}#1\end{pmatrix}}}
\newcommand{\mydet}[1]{\ensuremath{\begin{vmatrix}#1\end{vmatrix}}}
\numberwithin{equation}{subsection}
\makeatletter
\@addtoreset{figure}{problem}
\makeatother
\let\StandardTheFigure\thefigure
\let\vec\mathbf
\renewcommand{\thefigure}{\theproblem}
\def\putbox#1#2#3{\makebox[0in][l]{\makebox[#1][l]{}\raisebox{\baselineskip}[0in][0in]{\raisebox{#2}[0in][0in]{#3}}}}
     \def\rightbox#1{\makebox[0in][r]{#1}}
     \def\centbox#1{\makebox[0in]{#1}}
     \def\topbox#1{\raisebox{-\baselineskip}[0in][0in]{#1}}
     \def\midbox#1{\raisebox{-0.5\baselineskip}[0in][0in]{#1}}
\vspace{3cm}
\begin{center}
\huge Assignment 12\\
\large Shaik Zeeshan Ali\\
\large AI20MTECH11001\\
\end{center}
\begin{abstract}
This document is about the linear operator and minimal polynomials.
\end{abstract}
Download all python codes from 
\begin{lstlisting}
https://github.com/Zeeshan-IITH/IITH-EE5609/new/master/codes
\end{lstlisting}
and latex-tikz codes from 
\begin{lstlisting}
https://github.com/Zeeshan-IITH/IITH-EE5609
\end{lstlisting}
\section{problem}
Let $\vec{T}$ be a linear operator on $\mathbb{R}^2$, the matrix of which in the standard ordered basis is
\begin{align}
    \vec{A}=\myvec{1&-1\\2&2}
\end{align}
\begin{enumerate}
    \item Prove that the only subspaces of $\mathbb{R}^2$ invariant under $\vec{T}$ are $\mathbb{R}^2$ and the zero subspace.
    \item If $\vec{U}$ is the linear operator on $\mathbb{C}^2$, the matrix of which in the standard ordered basis is $\vec{A}$, show that $\vec{U}$ has 1-dimensional invariant subspaces.
\end{enumerate}
\section{Proof}
\renewcommand{\thetable}{1}
\begin{longtable}{|l|l|}
    \hline
        Definition & If $\vec{V}$ is a vector space over a field $\mathbb{F}$, a linear operator on $\vec{V}$ is a linear transformation \\
        Of linear &from $\vec{V}$ into $\vec{V}$. So for two vectors $\alpha$ and $\beta$ in $\vec{V}$, the transformation will be \\
        operator& $\vec{T}\brak{c\alpha+\beta}=c\brak{\vec{T}\alpha}+\vec{T}\beta$ where $\vec{T}\alpha$ and $\vec{T}\beta$ are in $\vec{V}$ and c in $\mathbb{F}$.\\
    \hline
    Invariant &If $\vec{W}$ is a subspace of $\vec{V}$, we say that $\vec{W}$ is invariant under $\vec{T}$ if for each vector $\alpha$\\
    subspaces &in $\vec{W}$ the vector $\vec{T}\alpha$ is in $\vec{W}$, i.e, if $\vec{T}\brak{\vec{W}}$ is contained in $\vec{W}$.\\
    \hline
    Given & $\vec{T}$ is a linear operator in $\mathbb{R}^2$, the matrix of which in the standard basis is\\
    & \qquad  \qquad \qquad \qquad \qquad$\vec{A}=\myvec{1&-1\\2&2}$\\
    & So by definition $\mathbb{R}^2$ is invariant under $\vec{T}$, since $\vec{T}$ is a linear operator.\\
    \hline
    \caption{construction}
    \label{tab:construction}
\end{longtable}
\renewcommand{\thetable}{2}
\begin{longtable}{|l|l|}
    \hline
    Characteristic & \qquad  \qquad \qquad \qquad $\det \brak{\vec{A}-\lambda \vec{I}}=\mydet{1-\lambda&-1\\2&2-\lambda}=\lambda^2-3\lambda+4$.\\
    equation of $\vec{A}$&\qquad  \qquad \qquad \qquad $\lambda_1=\frac{3+\sqrt{7}i}{2}$ and $\lambda_2=\frac{3-\sqrt{7}i}{2}$ are the eigenvalues of $\vec{A}$.\\
    \hline
    Proof for 1  & The null space of $\vec{T}$ can be calculated as\\
   & $\myvec{1&-1\\2&2}\myvec{x\\y}=0$, by augmenting we get $\myvec{1&-1&0\\2&2&0}$ \\
   & applying row reduction we get\\
   & $\myvec{1&-1&0\\2&2&0}\xleftrightarrow{R_2=R_2-2R_1}=\myvec{1&-1&0\\0&4&0}\xleftrightarrow{R_1=R_1+\frac{R_2}{4}}=\myvec{1&0&0\\0&4&0}\xleftrightarrow{R_2=\frac{R_2}{4}}\myvec{1&0&0\\0&1&0}$\\
   & Therefore the nullspace of $\vec{T}$ contains only the zero vector.\\
   &Assume that there is a 1-dimensional subspace that is invariant under $\vec{T}$,then \\
   & $\myvec{1&-1\\2&2}\myvec{x\\y}=c\myvec{x\\y}\implies\myvec{x\\y}$ is the eigen vector and c is eigen value.\\
   & Since the field of vector space is $\mathbb{R}$, there are no eigen values and hence no eigen vectors\\
   & Since there are no eigen values in the field $\mathbb{R}$, there are no 1-dimensional vectors which\\
   &can be invariant under $\vec{T}$.\\
   &Since the range of the linear operator is $\mathbb{R}^2$, the vector space $\mathbb{R}^2$ is invariant under $\vec{T}$.\\
   &And as the nullspace maps to zero vector and zero vector is invariant under $\vec{T}$, the only \\
   &subspaces that are invariant under $\vec{T}$ are the vector space $\mathbb{R}$ and the zero vector.\\
   \hline
   Proof for 2 & If $\vec{U}$ is the linear operator on $\mathbb{C}^2$, the matrix of which in the standard ordered basis is $\vec{A}$\\
   &The eigen vectors will be, for $\lambda_1=\frac{3+\sqrt{7}i}{2}$, the nullspace of $\vec{A}-\lambda\vec{I}$ will be the eigen vector.\\
   &\qquad \qquad $\myvec{\frac{-1-\sqrt{7}i}{2}&-1&0\\2&\frac{1-\sqrt{7}i}{2}&0}\xleftrightarrow{R_2=R_2-\frac{\brak{-1+\sqrt{7}i}R_1}{2}}$\myvec{\frac{-1-\sqrt{7}i}{2}&-1&0\\0&0&0}\\
   & Therefore one of the eigen vectors is $\vec{e_1}=\myvec{-1\\\frac{1+\sqrt{7}i}{2}}$\\
   & This eigen vector $\vec{e_1}$ is a subspace of $\mathbb{C}^2$.\\
   &\\
   \hline
   & Applying the linear operator $\vec{U}$ on $\vec{e_1}$, we get\\
   & $\vec{U\brak{ce_1}}=\vec{Ae_1}=c\myvec{1&-1\\2&2}\myvec{-1\\\frac{1+\sqrt{7}i}{2}}=c\frac{3+\sqrt{7}i}{2}\myvec{-1\\\frac{1+\sqrt{7}i}{2}}\implies\vec{U\brak{ce_1}}=c^{'}\vec{e_1}$\\
   &Therefore the vector space $\mathbb{C}^2$ has 1-dimensional invariant subspaces which are\\
   &the two subspaces along each eigen vectors containing the zero vector.\\
   \hline
    \caption{Proof}
    \label{tab:proof}
\end{longtable}
\end{document}

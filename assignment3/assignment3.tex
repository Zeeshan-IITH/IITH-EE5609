\documentclass[journal,12pt,twocolumn]{IEEEtran}
%
\usepackage{setspace}
\usepackage{gensymb}
%\doublespacing
\singlespacing

\usepackage[cmex10]{amsmath}
\usepackage{amsthm}
%\usepackage{iithtlc}
\usepackage{mathrsfs}
\usepackage{txfonts}
\usepackage{stfloats}
\usepackage{bm}
\usepackage{cite}
\usepackage{cases}
\usepackage{subfig}
%\usepackage{xtab}
\usepackage{longtable}
\usepackage{multirow}
%\usepackage{algorithm}
%\usepackage{algpseudocode}
\usepackage{enumitem}
\usepackage{mathtools}
\usepackage{steinmetz}
\usepackage{tikz}
\usepackage{circuitikz}
\usepackage{verbatim}
\usepackage{tfrupee}
\usepackage[breaklinks=true]{hyperref}
%\usepackage{stmaryrd}
\usepackage{tkz-euclide} % loads  TikZ and tkz-base
%\usetkzobj{all}
\usetikzlibrary{calc,math}
\usepackage{listings}
    \usepackage{color}                                            %%
    \usepackage{array}                                            %%
    \usepackage{longtable}                                        %%
    \usepackage{calc}                                             %%
    \usepackage{multirow}                                         %%
    \usepackage{hhline}                                           %%
    \usepackage{ifthen}                                           %%
  %optionally (for landscape tables embedded in another document): %%
    \usepackage{lscape}     
\usepackage{multicol}
\usepackage{chngcntr}
%\usepackage{enumerate}

%\usepackage{wasysym}
%\newcounter{MYtempeqncnt}
\DeclareMathOperator*{\Res}{Res}
%\renewcommand{\baselinestretch}{2}
\renewcommand\thesection{\arabic{section}}
\renewcommand\thesubsection{\thesection.\arabic{subsection}}
\renewcommand\thesubsubsection{\thesubsection.\arabic{subsubsection}}

\renewcommand\thesectiondis{\arabic{section}}
\renewcommand\thesubsectiondis{\thesectiondis.\arabic{subsection}}
\renewcommand\thesubsubsectiondis{\thesubsectiondis.\arabic{subsubsection}}

% correct bad hyphenation here
\hyphenation{op-tical net-works semi-conduc-tor}
\def\inputGnumericTable{}                                 %%

\lstset{
%language=C,
frame=single, 
breaklines=true,
columns=fullflexible
}
\newenvironment{amatrix}[1]{%
  \left(\begin{array}{@{}*{#1}{c}|c@{}}
}{%
  \end{array}\right)
}
\DeclarePairedDelimiter\abs{\lvert}{\rvert}%
\DeclarePairedDelimiter\norm{\lVert}{\rVert}%

% Swap the definition of \abs* and \norm*, so that \abs
% and \norm resizes the size of the brackets, and the 
% starred version does not.
\makeatletter
\let\oldabs\abs
\def\abs{\@ifstar{\oldabs}{\oldabs*}}
%
\let\oldnorm\norm
\def\norm{\@ifstar{\oldnorm}{\oldnorm*}}
\makeatother

\newtheorem{theorem}{Theorem}[section]
\newtheorem{problem}{Problem}
\newtheorem{proposition}{Proposition}[section]
\newtheorem{lemma}{Lemma}[section]
\newtheorem{corollary}[theorem]{Corollary}
\newtheorem{example}{Example}[section]
\newtheorem{definition}[problem]{Definition}
%\newtheorem{thm}{Theorem}[section] 
%\newtheorem{defn}[thm]{Definition}
%\newtheorem{algorithm}{Algorithm}[section]
%\newtheorem{cor}{Corollary}
\newcommand{\BEQA}{\begin{eqnarray}}
\newcommand{\EEQA}{\end{eqnarray}}
\newcommand{\define}{\stackrel{\triangle}{=}}
\bibliographystyle{IEEEtran}
%\bibliographystyle{ieeetr}
\providecommand{\mbf}{\mathbf}
\providecommand{\pr}[1]{\ensuremath{\Pr\left(#1\right)}}
\providecommand{\qfunc}[1]{\ensuremath{Q\left(#1\right)}}
\providecommand{\sbrak}[1]{\ensuremath{{}\left[#1\right]}}
\providecommand{\lsbrak}[1]{\ensuremath{{}\left[#1\right.}}
\providecommand{\rsbrak}[1]{\ensuremath{{}\left.#1\right]}}
\providecommand{\brak}[1]{\ensuremath{\left(#1\right)}}
\providecommand{\lbrak}[1]{\ensuremath{\left(#1\right.}}
\providecommand{\rbrak}[1]{\ensuremath{\left.#1\right)}}
\providecommand{\cbrak}[1]{\ensuremath{\left\{#1\right\}}}
\providecommand{\lcbrak}[1]{\ensuremath{\left\{#1\right.}}
\providecommand{\rcbrak}[1]{\ensuremath{\left.#1\right\}}}

\providecommand{\system}{\overset{\mathcal{H}}{ \longleftrightarrow}}
	%\newcommand{\solution}[2]{\textbf{Solution:}{#1}}
\newcommand{\solution}{\noindent \textbf{Solution: }}
\newcommand{\cosec}{\,\text{cosec}\,}
\providecommand{\dec}[2]{\ensuremath{\overset{#1}{\underset{#2}{\gtrless}}}}
\newcommand{\myvec}[1]{\ensuremath{\begin{pmatrix}#1\end{pmatrix}}}
\newcommand{\mydet}[1]{\ensuremath{\begin{vmatrix}#1\end{vmatrix}}}
%\numberwithin{equation}{section}
\numberwithin{equation}{subsection}
%\numberwithin{problem}{section}
%\numberwithin{definition}{section}
\makeatletter
\@addtoreset{figure}{problem}
\makeatother
\let\StandardTheFigure\thefigure
\let\vec\mathbf


\begin{document}

\begin{center}
\huge Assignment 3\\

\large Shaik Zeeshan Ali\\
\large AI20MTECH11001\\
\end{center}
\begin{abstract}
This document depicts a way to setup a matrix equation to find the fibonacci sequence.
\end{abstract}
Download all python codes from 
\begin{lstlisting}
https://github.com/Zeeshan-IITH/IITH-EE5609/new/master/codes
\end{lstlisting}

and latex-tikz codes from 
\begin{lstlisting}
https://github.com/Zeeshan-IITH/IITH-EE5609
\end{lstlisting}
\section{Problem}
For the matrix A=$\myvec{1&1&1 \\ 1&2&-3 \\ 2&-1&3}$.Show that \begin{align}
    A^3-6A^2+5A+11I=0\label{eq:1}
\end{align} and hence find $A^{-1}$.
\section{construction}
Given matrix is 
\begin{align}
    A=\myvec{1&1&1 \\ 1&2&-3 \\ 2&-1&3}\label{eq:2}
\end{align}
The characteristic polynomial of the matrix will be
\begin{align}
    \begin{bmatrix}A-\lambda\(I\)\end{bmatrix}=0\\
    =\begin{bmatrix}1-\lambda&1&1 \\ 1&2-\lambda&-3 \\ 2&-1&3-\lambda\end{bmatrix}\\
    =(1-\lambda)[(2-\lambda)(3-\lambda)-3]-1[(3-\lambda)\notag\\
    +6])+1[-1-2(2-\lambda)]\\
    =\lambda^3-6\lambda^2+5\lambda+11=0\label{eq:3}
\end{align}
According to cayley-hamilton theorem every matrix satisfies it's own characteristic equation. So the matrix $A$ satifies the equation \eqref{eq:1}
\begin{align}
    A^3-6A^2+5A+11I=0\label{eq:4}
\end{align}
\section{Inverse}
The matrix $A^2$=AA
\begin{align}
    A^2&=\myvec{1 & 1 & 1\\1 & 2 &-3\\2& -1 &3}\myvec{1 & 1 & 1\\1 & 2 &-3\\2& -1 &3}\notag\\
    &=\myvec{4&2&1\\-3&8&-14\\7&-3&14}
\end{align}
The matrix $A$ satisifes the characteristic equation,so
\begin{align}
    A^3-6A^2+5A+11I=0\notag
\end{align}
Multiplying with $A^-{^1}$ we get
\begin{align}
    A^2-6A+5I+11A^-{^1}=0\notag\\
    A^-{^1}=\frac{1}{11}\brak{6A-A^2-5I}\\
    A^-{^1}=\frac{1}{11}\brak{6\myvec{1 & 1 & 1\\1 & 2 &-3\\2& -1 &3}-\myvec{4&2&1\\-3&8&-14\\7&-3&14}\notag\\
    -5\myvec{1&0&0\\0&1&0\\0&0&1}}\\
    A^-{^1}=\frac{1}{11}\myvec{-3&4&5\\9&-1&-4\\5&-3&-1}
    \intertext{Therefore}
    A^-{^1}=\myvec{\frac{-3}{11}&\frac{4}{11}&\frac{5}{11}\\\frac{9}{11}&\frac{-1}{11}&\frac{-4}{11}\\\frac{5}{11}&\frac{-3}{11}&\frac{-1}{11}}
\end{align}
\end{document}

  \documentclass[journal,12pt,twocolumn]{IEEEtran}
%
\usepackage{setspace}
\usepackage{gensymb}
%\doublespacing
\singlespacing

\usepackage[cmex10]{amsmath}
\usepackage{amsthm}
%\usepackage{iithtlc}
\usepackage{mathrsfs}
\usepackage{txfonts}
\usepackage{stfloats}
\usepackage{bm}
\usepackage{cite}
\usepackage{cases}
\usepackage{subfig}
%\usepackage{xtab}
\usepackage{longtable}
\usepackage{multirow}
%\usepackage{algorithm}
%\usepackage{algpseudocode}
\usepackage{enumitem}
\usepackage{mathtools}
\usepackage{steinmetz}
\usepackage{tikz}
\usepackage{circuitikz}
\usepackage{verbatim}
\usepackage{tfrupee}
\usepackage[breaklinks=true]{hyperref}
%\usepackage{stmaryrd}
\usepackage{tkz-euclide} % loads  TikZ and tkz-base
%\usetkzobj{all}
\usetikzlibrary{calc,math}
\usepackage{listings}
    \usepackage{color}                                            %%
    \usepackage{array}                                            %%
    \usepackage{longtable}                                        %%
    \usepackage{calc}                                             %%
    \usepackage{multirow}                                         %%
    \usepackage{hhline}                                           %%
    \usepackage{ifthen}                                           %%
  %optionally (for landscape tables embedded in another document): %%
    \usepackage{lscape}     
\usepackage{multicol}
\usepackage{chngcntr}
%\usepackage{enumerate}

%\usepackage{wasysym}
%\newcounter{MYtempeqncnt}
\DeclareMathOperator*{\Res}{Res}
%\renewcommand{\baselinestretch}{2}
\renewcommand\thesection{\arabic{section}}
\renewcommand\thesubsection{\thesection.\arabic{subsection}}
\renewcommand\thesubsubsection{\thesubsection.\arabic{subsubsection}}

\renewcommand\thesectiondis{\arabic{section}}
\renewcommand\thesubsectiondis{\thesectiondis.\arabic{subsection}}
\renewcommand\thesubsubsectiondis{\thesubsectiondis.\arabic{subsubsection}}

% correct bad hyphenation here
\hyphenation{op-tical net-works semi-conduc-tor}
\def\inputGnumericTable{}                                 %%

\lstset{
%language=C,
frame=single, 
breaklines=true,
columns=fullflexible
}
\newenvironment{amatrix}[1]{%
  \left(\begin{array}{@{}*{#1}{c}|c@{}}
}{%
  \end{array}\right)
}
\DeclarePairedDelimiter\abs{\lvert}{\rvert}%
\DeclarePairedDelimiter\norm{\lVert}{\rVert}%

% Swap the definition of \abs* and \norm*, so that \abs
% and \norm resizes the size of the brackets, and the 
% starred version does not.
\makeatletter
\let\oldabs\abs
\def\abs{\@ifstar{\oldabs}{\oldabs*}}
%
\let\oldnorm\norm
\def\norm{\@ifstar{\oldnorm}{\oldnorm*}}
\makeatother

\newtheorem{theorem}{Theorem}[section]
\newtheorem{problem}{Problem}
\newtheorem{proposition}{Proposition}[section]
\newtheorem{lemma}{Lemma}[section]
\newtheorem{corollary}[theorem]{Corollary}
\newtheorem{example}{Example}[section]
\newtheorem{definition}[problem]{Definition}
%\newtheorem{thm}{Theorem}[section] 
%\newtheorem{defn}[thm]{Definition}
%\newtheorem{algorithm}{Algorithm}[section]
%\newtheorem{cor}{Corollary}
\newcommand{\BEQA}{\begin{eqnarray}}
\newcommand{\EEQA}{\end{eqnarray}}
\newcommand{\define}{\stackrel{\triangle}{=}}
\bibliographystyle{IEEEtran}
%\bibliographystyle{ieeetr}
\providecommand{\mbf}{\mathbf}
\providecommand{\pr}[1]{\ensuremath{\Pr\left(#1\right)}}
\providecommand{\qfunc}[1]{\ensuremath{Q\left(#1\right)}}
\providecommand{\sbrak}[1]{\ensuremath{{}\left[#1\right]}}
\providecommand{\lsbrak}[1]{\ensuremath{{}\left[#1\right.}}
\providecommand{\rsbrak}[1]{\ensuremath{{}\left.#1\right]}}
\providecommand{\brak}[1]{\ensuremath{\left(#1\right)}}
\providecommand{\lbrak}[1]{\ensuremath{\left(#1\right.}}
\providecommand{\rbrak}[1]{\ensuremath{\left.#1\right)}}
\providecommand{\cbrak}[1]{\ensuremath{\left\{#1\right\}}}
\providecommand{\lcbrak}[1]{\ensuremath{\left\{#1\right.}}
\providecommand{\rcbrak}[1]{\ensuremath{\left.#1\right\}}}

\providecommand{\system}{\overset{\mathcal{H}}{ \longleftrightarrow}}
	%\newcommand{\solution}[2]{\textbf{Solution:}{#1}}
\newcommand{\solution}{\noindent \textbf{Solution: }}
\newcommand{\cosec}{\,\text{cosec}\,}
\providecommand{\dec}[2]{\ensuremath{\overset{#1}{\underset{#2}{\gtrless}}}}
\newcommand{\myvec}[1]{\ensuremath{\begin{pmatrix}#1\end{pmatrix}}}
\newcommand{\mydet}[1]{\ensuremath{\begin{vmatrix}#1\end{vmatrix}}}
%\numberwithin{equation}{section}
\numberwithin{equation}{subsection}
%\numberwithin{problem}{section}
%\numberwithin{definition}{section}
\makeatletter
\@addtoreset{figure}{problem}
\makeatother
\let\StandardTheFigure\thefigure
\let\vec\mathbf
\usepackage{stackengine}
\newcommand\oast{\stackMath\mathbin{\stackinset{c}{0ex}{c}{0ex}{\ast}{\bigcirc}}}
\begin{document}

\begin{center}
\huge Circular Convolution\\

\large Shaik Zeeshan Ali\\
\large AI20MTECH11001\\
\end{center}
\begin{abstract}
This document tries to convert circular convolution in to matrix form
\end{abstract}
Download all python codes from 
\begin{lstlisting}
https://github.com/Zeeshan-IITH/IITH-EE5609/new/master/codes
\end{lstlisting}

and latex-tikz codes from 
\begin{lstlisting}
https://github.com/Zeeshan-IITH/IITH-EE5609
\end{lstlisting}
\section{problem}
A finite-length discrete-time signal is basically a sequence, say, $\left(x_0,…,x_m{_1}\right)$ which can be written as an m-length vector $vec{x}\in R^m$.\par
Given two periodic signals $\brak{x_0,…,x_n−1}$ and $\left(h_0,…,h_n−1\right)$, the circular convolution of the two signals is of length  $2n−1$, defined as
\begin{align}
    y\brak{t}=\brak{h\oast x}_t=\sum_{\tau=0}^{\tau=n-1} x_{\tau} h_\brak{t-\tau}mod n\label{eq:1}\\
    0\leq t < 2n-1\notag
\end{align}
\section{construction}
Circular convolution is a special case of convolution where the summation is truncated to a non-zero periodic interval and the actual signal is just a periodic repetition of the circular convolution.\par
The time period of the two discrete signals be $n$,then the resultant convolution is also discrete periodic signal with a time period $2n-1$.par
The signal $\vec{Y}$ contains $2n-1$ elements.
\begin{align}
    Y=\myvec{y_0\\y_1\\.\\.\\y_{2n-2}}
    \intertext{where}
    y_{t_0}=\sum_{\tau=0}^{\tau=n-1} x_{\tau} h_\brak{t_0-\tau}
    \intertext{Therefore}
    Y=\myvec{h_0x_0\\h_1x_0+h_0x_1\\h_0x_2+h_1x_1+h_2x_0\\.\\.\\h_{n-1}x_0+h_{n-2}x_1+...h_0x_{n-1}\\.\\h_{n-1}x_0+h_{n-2}x_1+...h_{0}x_{n-1}\\h_{n-1}x_1+h_{n-2}x_2+...h_{1}x_{n-1}\\.\\.\\h_{n-2}x_{n-1}+h_{n-1}x_{n-2}\\h_{n-1}x_{n-1}}
\end{align}
\section{Explanation}
Simplifying
\begin{align}
    Y=\myvec{h_0&0&0&.&.&0&0\\h_1&h_0&0&.&.&0&0\\h_2&h_1&h_0&.&.&0&0\\.&.&.&.&.&.&.\\h_{n-1}&h_{n-2}&h_{n-3}&.&.&h_1&h_0\\0&h_{n-1}&h_{n-2}&.&.&h_2&h_1\\0&0&h_{n-1}&.&.&h_3&h_2\\.&.&.&.&.&.&.\\0&0&0&.&.&0&h_{n-1}}\myvec{x_0\\x_1\\x_2.\\.\\.\\x_{n-1}}
\end{align}
Therefore we can write equation \eqref{eq:1} in matrix form as $\vec{Y}=\vec{H}\vec{X}$ where
\begin{align}
    \vec{H}=\myvec{h_0&0&0&.&.&0&0\\h_1&h_0&0&.&.&0&0\\h_2&h_1&h_0&.&.&0&0\\.&.&.&.&.&.&.\\h_{n-1}&h_{n-2}&h_{n-3}&.&.&h_1&h_0\\0&h_{n-1}&h_{n-2}&.&.&h_2&h_1\\0&0&h_{n-1}&.&.&h_3&h_2\\.&.&.&.&.&.&.\\0&0&0&.&.&0&h_{n-1}}
\end{align}
$\vec{H}$ is a matrix of dimension $(2n-1)\times n$.Since the actual signal is a periodic repitition of Each term of the signal $\vec{Y}$ can be written as
\begin{align}
    y_k=y_\brak{kmod(2n-1)}
\end{align}
Where each term $y_0,y_1...y_{2n-2}$ are derived from the circular convolution.
\end{document}

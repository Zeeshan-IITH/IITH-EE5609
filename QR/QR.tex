\documentclass[journal,12pt,twocolumn]{IEEEtran}
%
\usepackage{setspace}
\usepackage{gensymb}
%\doublespacing
\singlespacing

\usepackage[cmex10]{amsmath}
\usepackage{amsthm}
%\usepackage{iithtlc}
\usepackage{mathrsfs}
\usepackage{txfonts}
\usepackage{stfloats}
\usepackage{bm}
\usepackage{cite}
\usepackage{cases}
\usepackage{subfig}
%\usepackage{xtab}
\usepackage{longtable}
\usepackage{multirow}
%\usepackage{algorithm}
%\usepackage{algpseudocode}
\usepackage{enumitem}
\usepackage{mathtools}
\usepackage{steinmetz}
\usepackage{tikz}
\usepackage{circuitikz}
\usepackage{verbatim}
\usepackage{tfrupee}
\usepackage[breaklinks=true]{hyperref}
%\usepackage{stmaryrd}
\usepackage{tkz-euclide} % loads  TikZ and tkz-base
%\usetkzobj{all}
\usetikzlibrary{calc,math}
\usepackage{listings}
    \usepackage{color}                                            %%
    \usepackage{array}                                            %%
    \usepackage{longtable}                                        %%
    \usepackage{calc}                                             %%
    \usepackage{multirow}                                         %%
    \usepackage{hhline}                                           %%
    \usepackage{ifthen}                                           %%
  %optionally (for landscape tables embedded in another document): %%
    \usepackage{lscape}     
\usepackage{multicol}
\usepackage{chngcntr}
%\usepackage{enumerate}

%\usepackage{wasysym}
%\newcounter{MYtempeqncnt}
\DeclareMathOperator*{\Res}{Res}
%\renewcommand{\baselinestretch}{2}
\renewcommand\thesection{\arabic{section}}
\renewcommand\thesubsection{\thesection.\arabic{subsection}}
\renewcommand\thesubsubsection{\thesubsection.\arabic{subsubsection}}

\renewcommand\thesectiondis{\arabic{section}}
\renewcommand\thesubsectiondis{\thesectiondis.\arabic{subsection}}
\renewcommand\thesubsubsectiondis{\thesubsectiondis.\arabic{subsubsection}}

% correct bad hyphenation here
\hyphenation{op-tical net-works semi-conduc-tor}
\def\inputGnumericTable{}                                 %%

\lstset{
%language=C,
frame=single, 
breaklines=true,
columns=fullflexible
}
\newenvironment{amatrix}[1]{%
  \left(\begin{array}{@{}*{#1}{c}|c@{}}
}{%
  \end{array}\right)
}
\DeclarePairedDelimiter\abs{\lvert}{\rvert}%
\DeclarePairedDelimiter\norm{\lVert}{\rVert}%

% Swap the definition of \abs* and \norm*, so that \abs
% and \norm resizes the size of the brackets, and the 
% starred version does not.
\makeatletter
\let\oldabs\abs
\def\abs{\@ifstar{\oldabs}{\oldabs*}}
%
\let\oldnorm\norm
\def\norm{\@ifstar{\oldnorm}{\oldnorm*}}
\makeatother

\newtheorem{theorem}{Theorem}[section]
\newtheorem{problem}{Problem}
\newtheorem{proposition}{Proposition}[section]
\newtheorem{lemma}{Lemma}[section]
\newtheorem{corollary}[theorem]{Corollary}
\newtheorem{example}{Example}[section]
\newtheorem{definition}[problem]{Definition}
%\newtheorem{thm}{Theorem}[section] 
%\newtheorem{defn}[thm]{Definition}
%\newtheorem{algorithm}{Algorithm}[section]
%\newtheorem{cor}{Corollary}
\newcommand{\BEQA}{\begin{eqnarray}}
\newcommand{\EEQA}{\end{eqnarray}}
\newcommand{\define}{\stackrel{\triangle}{=}}
\bibliographystyle{IEEEtran}
%\bibliographystyle{ieeetr}
\providecommand{\mbf}{\mathbf}
\providecommand{\pr}[1]{\ensuremath{\Pr\left(#1\right)}}
\providecommand{\qfunc}[1]{\ensuremath{Q\left(#1\right)}}
\providecommand{\sbrak}[1]{\ensuremath{{}\left[#1\right]}}
\providecommand{\lsbrak}[1]{\ensuremath{{}\left[#1\right.}}
\providecommand{\rsbrak}[1]{\ensuremath{{}\left.#1\right]}}
\providecommand{\brak}[1]{\ensuremath{\left(#1\right)}}
\providecommand{\lbrak}[1]{\ensuremath{\left(#1\right.}}
\providecommand{\rbrak}[1]{\ensuremath{\left.#1\right)}}
\providecommand{\cbrak}[1]{\ensuremath{\left\{#1\right\}}}
\providecommand{\lcbrak}[1]{\ensuremath{\left\{#1\right.}}
\providecommand{\rcbrak}[1]{\ensuremath{\left.#1\right\}}}

\providecommand{\system}{\overset{\mathcal{H}}{ \longleftrightarrow}}
	%\newcommand{\solution}[2]{\textbf{Solution:}{#1}}
\newcommand{\solution}{\noindent \textbf{Solution: }}
\newcommand{\cosec}{\,\text{cosec}\,}
\providecommand{\dec}[2]{\ensuremath{\overset{#1}{\underset{#2}{\gtrless}}}}
\newcommand{\myvec}[1]{\ensuremath{\begin{pmatrix}#1\end{pmatrix}}}
\newcommand{\mydet}[1]{\ensuremath{\begin{vmatrix}#1\end{vmatrix}}}
%\numberwithin{equation}{section}
\numberwithin{equation}{subsection}
%\numberwithin{problem}{section}
%\numberwithin{definition}{section}
\makeatletter
\@addtoreset{figure}{problem}
\makeatother
\let\StandardTheFigure\thefigure
\let\vec\mathbf


\begin{document}

\begin{center}
\huge QR decomposition\\

\large Shaik Zeeshan Ali\\
\large AI20MTECH11001\\
\end{center}
\begin{abstract}
This document analyzes QR decomposition of a non-singular square matrix
\end{abstract}
Download all python codes from 
\begin{lstlisting}
https://github.com/Zeeshan-IITH/IITH-EE5609/new/master/codes
\end{lstlisting}

and latex-tikz codes from 
\begin{lstlisting}
https://github.com/Zeeshan-IITH/IITH-EE5609
\end{lstlisting}
\section{problem}
What is the QR decomposition of a non-singular square matrix whose columns are mutually orthogonal.
\section{construction}
The purpose of QR decomposition can be thought of as constructing a basis for the column space of a non-singular matrix.So the matrix Q represents the orthonormal basis of the columnn space and the matrix R represents the weights that each of the orthonormal basis vectors carry for each column vector.Let $\vec{A}$ be a square matrix of order $n\times n$ whose column vectors are mutually orthogonal.
\begin{align}
    \vec{A}=\myvec{c_1& c_2 & c_3 & .& .&.&c_n}\\
    \intertext{where}
    c_i^Tc_j=0\notag\\
    i\neq j\label{eq:1}
\end{align}
where every column vector is of dimension $n\times 1$.So the column space of the matrix $\vec{A}$ is a set of $n$ linearly independent vectors\par
The orthogonal matrix Q will be 
\begin{align}
    \vec{Q}=\myvec{q_1&q_2&.&.&.&q_n}\\
    \vec{Q}\vec{Q}^T=\vec{Q}^T\vec{Q}=\vec{I}
\end{align}
where $q_i$ is a column vector of dimension $n\times1$ and $q_i^Tq_i=1$ and $q_i^Tq_j=0$ if $i\neq j$.\par
we can define an upper triangular matrix R as
\begin{align}
    \vec{R}=\myvec{r_{11}&r_{12}&.&.&r_{1n}\\0&r_{22}&.&.&r_{2n}\\.&.&.&.&.\\0&0&.&.&r_{nn}}
\end{align}
The matrix $\vec{A}$ can be written in terms of $\vec{Q}$ and $\vec{R}$ as
\begin{align}
    \vec{A}=\vec{Q}\vec{R}
\end{align}
\section{Explanation}
Because the column vectors of $\vec{A}$ are orthogonal to each other
\begin{align}
    \vec{A}^T\vec{A}&=\myvec{c_1& c_2 & .& .&.&c_n}^T\myvec{c_1& c_2  & .& .&.&c_n}\\
    &=\myvec{c_1^T\\ c_2^T\\ c_3^T \\ .\\ .\\.\\c_n^T}\myvec{c_1& c_2 & c_3 & .& .&.&c_n}\\
    &=\myvec{\norm{c_1}^2& 0 & .& .&.&0\\0&\norm{c_2}^2&.&.&.&0\\.&.&.&.&.&.\\0&0&.&.&.&\norm{c_n}^2}\label{eq:2}
\end{align}
which is a diagonal matrix because every column vector is mutually orthogonal.\par
\begin{align}
    \vec{A}^T\vec{A}&=\brak{\vec{Q}\vec{R}}^T\vec{Q}\vec{R}\\
    &=\vec{R}^T\vec{Q}^T\vec{Q}\vec{R}\\
    &=\vec{R}^T\vec{I}\vec{R}\\
    &=\vec{R}^T\vec{R}\\
    &=\myvec{r_{11}^2& r_{11}r_{12} & .& .&.&r_{11}r_{1n}\\r_{11}r_{12}&r_{22}^2&.&.&.&r_{22}r_{2n}\\.&.&.&.&.&.\\r_{11}r_{1n}&r_{22}r_{2n}&.&.&.&r_{nn}^2}\label{eq:3}
\end{align}
Comparing equations \eqref{eq:2} and \eqref{eq:3},we get 
\begin{align}
    r_{ii}=\norm{c_i}\\
    r_{ij}=0
\end{align}
The matrix $\vec{R}$ will be
\begin{align}
    \vec{R}=\myvec{\norm{c_1}& 0 & .& .&.&0\\0&\norm{c_2}&.&.&.&0\\.&.&.&.&.&.\\0&0&.&.&.&\norm{c_n}}
\end{align}
Therefore the orthonormal vectors of  $\vec{Q}$ will be the normalized column vectors of the matrix $\vec{A}$.
\begin{align}
    \vec{Q}=\myvec{\frac{c_1}{\norm{c_1}}&\frac{c_2}{\norm{c_2}}&.&.&\frac{c_n}{\norm{c_n}}}
\end{align}
\end{document}

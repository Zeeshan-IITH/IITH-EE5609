\documentclass[journal,12pt,twocolumn]{IEEEtran}
%
\usepackage{setspace}
\usepackage{gensymb}
%\doublespacing
\singlespacing

\usepackage[cmex10]{amsmath}
\usepackage{amsthm}
%\usepackage{iithtlc}
\usepackage{mathrsfs}
\usepackage{txfonts}
\usepackage{stfloats}
\usepackage{bm}
\usepackage{cite}
\usepackage{cases}
\usepackage{subfig}
%\usepackage{xtab}
\usepackage{longtable}
\usepackage{multirow}
%\usepackage{algorithm}
%\usepackage{algpseudocode}
\usepackage{enumitem}
\usepackage{mathtools}
\usepackage{steinmetz}
\usepackage{tikz}
\usepackage{circuitikz}
\usepackage{verbatim}
\usepackage{tfrupee}
\usepackage[breaklinks=true]{hyperref}
%\usepackage{stmaryrd}
\usepackage{tkz-euclide} % loads  TikZ and tkz-base
%\usetkzobj{all}
\usetikzlibrary{calc,math}
\usepackage{listings}
    \usepackage{color}                                            %%
    \usepackage{array}                                            %%
    \usepackage{longtable}                                        %%
    \usepackage{calc}                                             %%
    \usepackage{multirow}                                         %%
    \usepackage{hhline}                                           %%
    \usepackage{ifthen}                                           %%
  %optionally (for landscape tables embedded in another document): %%
    \usepackage{lscape}     
\usepackage{multicol}
\usepackage{chngcntr}
%\usepackage{enumerate}

%\usepackage{wasysym}
%\newcounter{MYtempeqncnt}
\DeclareMathOperator*{\Res}{Res}
%\renewcommand{\baselinestretch}{2}
\renewcommand\thesection{\arabic{section}}
\renewcommand\thesubsection{\thesection.\arabic{subsection}}
\renewcommand\thesubsubsection{\thesubsection.\arabic{subsubsection}}

\renewcommand\thesectiondis{\arabic{section}}
\renewcommand\thesubsectiondis{\thesectiondis.\arabic{subsection}}
\renewcommand\thesubsubsectiondis{\thesubsectiondis.\arabic{subsubsection}}

% correct bad hyphenation here
\hyphenation{op-tical net-works semi-conduc-tor}
\def\inputGnumericTable{}                                 %%

\lstset{
%language=C,
frame=single, 
breaklines=true,
columns=fullflexible
}
\newenvironment{amatrix}[1]{%
  \left(\begin{array}{@{}*{#1}{c}|c@{}}
}{%
  \end{array}\right)
}
\DeclarePairedDelimiter\abs{\lvert}{\rvert}%
\DeclarePairedDelimiter\norm{\lVert}{\rVert}%

% Swap the definition of \abs* and \norm*, so that \abs
% and \norm resizes the size of the brackets, and the 
% starred version does not.
\makeatletter
\let\oldabs\abs
\def\abs{\@ifstar{\oldabs}{\oldabs*}}
%
\let\oldnorm\norm
\def\norm{\@ifstar{\oldnorm}{\oldnorm*}}
\makeatother

\newtheorem{theorem}{Theorem}[section]
\newtheorem{problem}{Problem}
\newtheorem{proposition}{Proposition}[section]
\newtheorem{lemma}{Lemma}[section]
\newtheorem{corollary}[theorem]{Corollary}
\newtheorem{example}{Example}[section]
\newtheorem{definition}[problem]{Definition}
%\newtheorem{thm}{Theorem}[section] 
%\newtheorem{defn}[thm]{Definition}
%\newtheorem{algorithm}{Algorithm}[section]
%\newtheorem{cor}{Corollary}
\newcommand{\BEQA}{\begin{eqnarray}}
\newcommand{\EEQA}{\end{eqnarray}}
\newcommand{\define}{\stackrel{\triangle}{=}}
\bibliographystyle{IEEEtran}
%\bibliographystyle{ieeetr}
\providecommand{\mbf}{\mathbf}
\providecommand{\pr}[1]{\ensuremath{\Pr\left(#1\right)}}
\providecommand{\qfunc}[1]{\ensuremath{Q\left(#1\right)}}
\providecommand{\sbrak}[1]{\ensuremath{{}\left[#1\right]}}
\providecommand{\lsbrak}[1]{\ensuremath{{}\left[#1\right.}}
\providecommand{\rsbrak}[1]{\ensuremath{{}\left.#1\right]}}
\providecommand{\brak}[1]{\ensuremath{\left(#1\right)}}
\providecommand{\lbrak}[1]{\ensuremath{\left(#1\right.}}
\providecommand{\rbrak}[1]{\ensuremath{\left.#1\right)}}
\providecommand{\cbrak}[1]{\ensuremath{\left\{#1\right\}}}
\providecommand{\lcbrak}[1]{\ensuremath{\left\{#1\right.}}
\providecommand{\rcbrak}[1]{\ensuremath{\left.#1\right\}}}

\providecommand{\system}{\overset{\mathcal{H}}{ \longleftrightarrow}}
	%\newcommand{\solution}[2]{\textbf{Solution:}{#1}}
\newcommand{\solution}{\noindent \textbf{Solution: }}
\newcommand{\cosec}{\,\text{cosec}\,}
\providecommand{\dec}[2]{\ensuremath{\overset{#1}{\underset{#2}{\gtrless}}}}
\newcommand{\myvec}[1]{\ensuremath{\begin{pmatrix}#1\end{pmatrix}}}
\newcommand{\mydet}[1]{\ensuremath{\begin{vmatrix}#1\end{vmatrix}}}
%\numberwithin{equation}{section}
\numberwithin{equation}{subsection}
%\numberwithin{problem}{section}
%\numberwithin{definition}{section}
\makeatletter
\@addtoreset{figure}{problem}
\makeatother
\let\StandardTheFigure\thefigure
\let\vec\mathbf


\begin{document}

\begin{center}
\huge Fibonacci numbers\\

\large Shaik Zeeshan Ali\\
\large AI20MTECH11001\\
\end{center}
\begin{abstract}
This document depicts a way to setup a matrix equation to find the fibonacci sequence.
\end{abstract}
Download all python codes from 
\begin{lstlisting}
https://github.com/Zeeshan-IITH/IITH-EE5609/new/master/codes
\end{lstlisting}

and latex-tikz codes from 
\begin{lstlisting}
https://github.com/Zeeshan-IITH/IITH-EE5609
\end{lstlisting}
\section{Problem}
Given a $kxk$ matrix $\vec{A}$,find the powers of $\vec{A}^n$ within $O(log n)$ time.
\section{Construction}
For the sake of simplicity we will be calculating the powers given $n=2^m$, where $n$ is much larger than $k$.The required result will be of the form $\vec{A}^1,\vec{A}^2,\vec{A}^4,\vec{A}^8,\vec{A}^1{^6}$..
\begin{align}
    \intertext{The first matrix multiplication will be}\notag\\
    \vec{A}^2=\vec{A}\vec{A}
    \intertext{Instead of using repeated multiplication by $\vec{A}$,we can use the previous result and square it to be closer to the result using less complutations}\notag\\
    \vec{A}^4=\vec{A}^2\vec{A}^2\\
    \vec{A}^8=\vec{A}^4\vec{A}^4\\
    \vec{A}^1{^6}=\vec{A}^8\vec{A}^8\\
    \intertext{So $\vec{A}^{2^m}$ need only m products of the resultant matrix from the previous computation.Since $m=log_2(n)$, the result can be computed in $O(log_2n)$ time.}\notag
\end{align}
As a more general case any number $n$ can be represented as a sum of powers of $2$, just like binary numbers are represented with the radix 2 i.e when we represent $n$ in binary form we get
\begin{align}
    n=b_kb_k{_-}{_1}b_k{_-}{_2}...b_0
    \intertext{where}\notag
    k=\log_2 n
\end{align}
because we require that many number of bits to represent $n$.\par
Now to calculate
\begin{align}
    \bm{A}^n\notag \intertext{we can use}\notag\\
    \bm{A}^n=b_k\bm{A}^{2^k}+b_k{_-}{_1}\bm{A}^{2^k{^-}{^1}}+b_k{_-}{_2}\bm{A}^{2^k{^-}{^2}}...+b_2\bm{A}^4+b_1\bm{A}^2+b_0\bm{A}
\end{align}
Now each of the $\bm{A}^{2^k}$ can be calculated by squaring the previous $\bm{A}^{2^k{^-}{^1}}$.Neglecting matrix addition as it only takes $m^2$ time to compute when compared to $m^3$ time of each multiplication,where m is the order of the square matrix, we can say that since $\bm{A}^{2^k}$ calculation takes $O(\log_2 n)$ time,the whole computation $\bm{A}^n$ takes $O(\log_2 n)$.
\section{Fibonacci}
Consider the special matrix which begins with fibonacci numbers in it
\begin{align}
    \vec{F}=\myvec{1 & 1\\1 & 0}
    \intertext{This matrix has the fibonaci numbers as its elements}
    \vec{F^2}=\myvec{2 & 1\\1 & 0}\\
    \vec{F^3}=\myvec{3 & 2\\2 & 1}
    \intertext{Therefore the $n^t{^h}$ fibonacci number $F_n$ is the element in the first row,first column of $\vec{F^n{^-}{^1}}$,where $n\geq2$.The general form will be}
    \vec{F^n{^-}{^1}}=\myvec{F_n & F_n{_-{_1}}\\F_n{_-{_2}} & F_n{_-}{_3}}
\end{align}
\end{document}

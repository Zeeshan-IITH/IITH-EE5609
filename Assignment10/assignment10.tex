\documentclass[journal,12pt,twocolumn]{IEEEtran}
%
\usepackage{setspace}
\usepackage{textcomp}
\usepackage{gensymb}
%\doublespacing
\singlespacing

\usepackage[cmex10]{amsmath}
\usepackage{amsthm}
%\usepackage{iithtlc}
\usepackage{mathrsfs}
\usepackage{txfonts}
\usepackage{stfloats}
\usepackage{bm}
\usepackage{cite}
\usepackage{cases}
\usepackage{subfig}
%\usepackage{xtab}
\usepackage{longtable}
\usepackage{multirow}
%\usepackage{algorithm}
%\usepackage{algpseudocode}
\usepackage{enumitem}
\usepackage{mathtools}
\usepackage{steinmetz}
\usepackage{tikz}
\usepackage{circuitikz}
\usepackage{verbatim}
\usepackage{tfrupee}
\usepackage[breaklinks=true]{hyperref}
%\usepackage{stmaryrd}
\usepackage{tkz-euclide} % loads  TikZ and tkz-base
%\usetkzobj{all}
\usetikzlibrary{calc,math}
\usepackage{listings}
    \usepackage{color}                                            %%
    \usepackage{array}                                            %%
    \usepackage{longtable}                                        %%
    \usepackage{calc}                                             %%
    \usepackage{multirow}                                         %%
    \usepackage{hhline}                                           %%
    \usepackage{ifthen}                                           %%
  %optionally (for landscape tables embedded in another document): %%
    \usepackage{lscape}     
\usepackage{multicol}
\usepackage{chngcntr}
%\usepackage{enumerate}

%\usepackage{wasysym}
%\newcounter{MYtempeqncnt}
\DeclareMathOperator*{\Res}{Res}
%\renewcommand{\baselinestretch}{2}
\renewcommand\thesection{\arabic{section}}
\renewcommand\thesubsection{\thesection.\arabic{subsection}}
\renewcommand\thesubsubsection{\thesubsection.\arabic{subsubsection}}

\renewcommand\thesectiondis{\arabic{section}}
\renewcommand\thesubsectiondis{\thesectiondis.\arabic{subsection}}
\renewcommand\thesubsubsectiondis{\thesubsectiondis.\arabic{subsubsection}}

% correct bad hyphenation here
\hyphenation{op-tical net-works semi-conduc-tor}
\def\inputGnumericTable{}                                 %%

\lstset{
%language=C,
frame=single, 
breaklines=true,
columns=fullflexible
}
\newenvironment{amatrix}[1]{%
  \left(\begin{array}{@{}*{#1}{c}|c@{}}
}{%
  \end{array}\right)
}
\DeclarePairedDelimiter\abs{\lvert}{\rvert}%
\DeclarePairedDelimiter\norm{\lVert}{\rVert}%

% Swap the definition of \abs* and \norm*, so that \abs
% and \norm resizes the size of the brackets, and the 
% starred version does not.
\makeatletter
\let\oldabs\abs
\def\abs{\@ifstar{\oldabs}{\oldabs*}}
%
\let\oldnorm\norm
\def\norm{\@ifstar{\oldnorm}{\oldnorm*}}
\makeatother

\newtheorem{theorem}{Theorem}[section]
\newtheorem{problem}{Problem}
\newtheorem{proposition}{Proposition}[section]
\newtheorem{lemma}{Lemma}[section]
\newtheorem{corollary}[theorem]{Corollary}
\newtheorem{example}{Example}[section]
\newtheorem{definition}[problem]{Definition}
%\newtheorem{thm}{Theorem}[section] 
%\newtheorem{defn}[thm]{Definition}
%\newtheorem{algorithm}{Algorithm}[section]
%\newtheorem{cor}{Corollary}
\newcommand{\BEQA}{\begin{eqnarray}}
\newcommand{\EEQA}{\end{eqnarray}}
\newcommand{\define}{\stackrel{\triangle}{=}}
\bibliographystyle{IEEEtran}
%\bibliographystyle{ieeetr}
\providecommand{\mbf}{\mathbf}
\providecommand{\pr}[1]{\ensuremath{\Pr\left(#1\right)}}
\providecommand{\qfunc}[1]{\ensuremath{Q\left(#1\right)}}
\providecommand{\sbrak}[1]{\ensuremath{{}\left[#1\right]}}
\providecommand{\lsbrak}[1]{\ensuremath{{}\left[#1\right.}}
\providecommand{\rsbrak}[1]{\ensuremath{{}\left.#1\right]}}
\providecommand{\brak}[1]{\ensuremath{\left(#1\right)}}
\providecommand{\lbrak}[1]{\ensuremath{\left(#1\right.}}
\providecommand{\rbrak}[1]{\ensuremath{\left.#1\right)}}
\providecommand{\cbrak}[1]{\ensuremath{\left\{#1\right\}}}
\providecommand{\lcbrak}[1]{\ensuremath{\left\{#1\right.}}
\providecommand{\rcbrak}[1]{\ensuremath{\left.#1\right\}}}
\providecommand{\system}{\overset{\mathcal{H}}{ \longleftrightarrow}}
	%\newcommand{\solution}[2]{\textbf{Solution:}{#1}}
\newcommand{\solution}{\noindent \textbf{Solution: }}
\newcommand{\cosec}{\,\text{cosec}\,}
\providecommand{\dec}[2]{\ensuremath{\overset{#1}{\underset{#2}{\gtrless}}}}
\newcommand{\myvec}[1]{\ensuremath{\begin{pmatrix}#1\end{pmatrix}}}
\newcommand{\mydet}[1]{\ensuremath{\begin{vmatrix}#1\end{vmatrix}}}
%\numberwithin{equation}{section}
\numberwithin{equation}{subsection}
%\numberwithin{problem}{section}
%\numberwithin{definition}{section}
\makeatletter
\@addtoreset{figure}{problem}
\makeatother
\let\StandardTheFigure\thefigure
\let\vec\mathbf
\usepackage{mathtools, nccmath}
\usepackage{longtable}
\renewcommand{\arraystretch}{1.5}
\usepackage{afterpage}
\newcommand\myemptypage{
	\null
	\thispagestyle{empty}
	\addtocounter{page}{-1}
	\newpage
}

\begin{document}

\begin{center}
\huge Assignment 10\\

\large Shaik Zeeshan Ali\\
\large AI20MTECH11001\\
\end{center}
\begin{abstract}
This document is about inverse of the given matrices.
\end{abstract}
Download all python codes from 
\begin{lstlisting}
https://github.com/Zeeshan-IITH/IITH-EE5609/new/master/codes
\end{lstlisting}

and latex-tikz codes from 
\begin{lstlisting}
https://github.com/Zeeshan-IITH/IITH-EE5609
\end{lstlisting}
\section{problem}
Let $\vec{A}$ and $\vec{B}$ be $n\times n$ matrices over the field $F$.Prove that if $\brak{\vec{I}-\vec{A}\vec{B}}$ is invertibe
\begin{enumerate}
    \item $\brak{\vec{I}-\vec{B}\vec{A}}$ is invertible and
    \item $\brak{\vec{I}-\vec{B}\vec{A}}^{-1}=\vec{I}+\vec{B}\brak{\vec{I}-\vec{A}\vec{B}}^{-1}\vec{A}$
\end{enumerate}
\section{PROOF}
\begin{longtable}{|c|c|}
\caption{PROOF}
\hline
\text{Invertible} & \text{ A matrix $\vec{M}$ is invertible if it is non-singular i.e. the null space of $\vec{M}$ contains only}\\
& \text{zero vector.If $\vec{x}$ is a vector such that $\vec{M}\vec{x}=0\implies \vec{x}=0$}\\
\hline
\text{Proof for 1} & \text{Consider a vecor $\vec{y}$ such that $\brak{\vec{I}-\vec{B}\vec{A}}\vec{y}=0$}\\
& \text{$\brak{\vec{I}-\vec{B}\vec{A}}\vec{y}=0\implies\vec{y}=\vec{BA}\vec{y}$}\\
& \text{$\vec{A}\vec{y}=\vec{ABA}\vec{y}\implies\brak{\vec{I}-\vec{A}\vec{B}}\vec{A}\vec{y}=0$}\\
& \text{since the matrix $\brak{\vec{I}-\vec{A}\vec{B}}$ is invertible,$\vec{A}\vec{y}=0$}\\
& \text{$\vec{y}=\vec{B}\brak{\vec{Ay}}\implies\vec{y}=0$}\\
& \text{Hence the matrix $\brak{\vec{I}-\vec{B}\vec{A}}$ is invertible.}\\
\hline
\text{Observation} & \text{Let $\vec{C}=\brak{\vec{I}-\vec{AB}}^{-1}$,then}\\
& \text{$\brak{\vec{I}-\vec{BA}}\brak{\vec{BCA}}=\vec{BCA}-\vec{BABCA}=\vec{B}\brak{\vec{I}-\vec{AB}}\vec{CA}=\vec{BC^{-1}CA}=\vec{BA}$}\\
& \text{$\vec{\brak{\vec{I}-\vec{BA}}\brak{\vec{BCA}}=\vec{BA}}$}
\hline
\text{Proof for 2} & \text{Let us consider the product $\vec{\brak{I-BA}\brak{BCA+I}}$}\\
& \text{$\vec{\brak{I-BA}\brak{BCA+I}}=\vec{\brak{\vec{I}-\vec{BA}}\brak{\vec{BCA}}}+\brak{\vec{I-BA}}$}\\
& \text{$=\vec{BA+I-BA}=\vec{I}$}\\
& \text{$\brak{\vec{I-BA}}^{-1}=\vec{I+B\brak{I-AB}^{-1}A}$}\\
& \text{Hence proved.}
\hline
\end{longtable}
\end{document}

\documentclass[journal,12pt,twocolumn]{IEEEtran}
%
\usepackage{setspace}
\usepackage{gensymb}
%\doublespacing
\singlespacing

\usepackage[cmex10]{amsmath}
\usepackage{amsthm}
%\usepackage{iithtlc}
\usepackage{mathrsfs}
\usepackage{txfonts}
\usepackage{stfloats}
\usepackage{bm}
\usepackage{cite}
\usepackage{cases}
\usepackage{subfig}
%\usepackage{xtab}
\usepackage{longtable}
\usepackage{multirow}
%\usepackage{algorithm}
%\usepackage{algpseudocode}
\usepackage{enumitem}
\usepackage{mathtools}
\usepackage{steinmetz}
\usepackage{tikz}
\usepackage{circuitikz}
\usepackage{verbatim}
\usepackage{tfrupee}
\usepackage[breaklinks=true]{hyperref}
%\usepackage{stmaryrd}
\usepackage{tkz-euclide} % loads  TikZ and tkz-base
%\usetkzobj{all}
\usetikzlibrary{calc,math}
\usepackage{listings}
    \usepackage{color}                                            %%
    \usepackage{array}                                            %%
    \usepackage{longtable}                                        %%
    \usepackage{calc}                                             %%
    \usepackage{multirow}                                         %%
    \usepackage{hhline}                                           %%
    \usepackage{ifthen}                                           %%
  %optionally (for landscape tables embedded in another document): %%
    \usepackage{lscape}     
\usepackage{multicol}
\usepackage{chngcntr}
%\usepackage{enumerate}

%\usepackage{wasysym}
%\newcounter{MYtempeqncnt}
\DeclareMathOperator*{\Res}{Res}
%\renewcommand{\baselinestretch}{2}
\renewcommand\thesection{\arabic{section}}
\renewcommand\thesubsection{\thesection.\arabic{subsection}}
\renewcommand\thesubsubsection{\thesubsection.\arabic{subsubsection}}

\renewcommand\thesectiondis{\arabic{section}}
\renewcommand\thesubsectiondis{\thesectiondis.\arabic{subsection}}
\renewcommand\thesubsubsectiondis{\thesubsectiondis.\arabic{subsubsection}}

% correct bad hyphenation here
\hyphenation{op-tical net-works semi-conduc-tor}
\def\inputGnumericTable{}                                 %%

\lstset{
%language=C,
frame=single, 
breaklines=true,
columns=fullflexible
}
\newenvironment{amatrix}[1]{%
  \left(\begin{array}{@{}*{#1}{c}|c@{}}
}{%
  \end{array}\right)
}
\DeclarePairedDelimiter\abs{\lvert}{\rvert}%
\DeclarePairedDelimiter\norm{\lVert}{\rVert}%

% Swap the definition of \abs* and \norm*, so that \abs
% and \norm resizes the size of the brackets, and the 
% starred version does not.
\makeatletter
\let\oldabs\abs
\def\abs{\@ifstar{\oldabs}{\oldabs*}}
%
\let\oldnorm\norm
\def\norm{\@ifstar{\oldnorm}{\oldnorm*}}
\makeatother

\newtheorem{theorem}{Theorem}[section]
\newtheorem{problem}{Problem}
\newtheorem{proposition}{Proposition}[section]
\newtheorem{lemma}{Lemma}[section]
\newtheorem{corollary}[theorem]{Corollary}
\newtheorem{example}{Example}[section]
\newtheorem{definition}[problem]{Definition}
%\newtheorem{thm}{Theorem}[section] 
%\newtheorem{defn}[thm]{Definition}
%\newtheorem{algorithm}{Algorithm}[section]
%\newtheorem{cor}{Corollary}
\newcommand{\BEQA}{\begin{eqnarray}}
\newcommand{\EEQA}{\end{eqnarray}}
\newcommand{\define}{\stackrel{\triangle}{=}}
\bibliographystyle{IEEEtran}
%\bibliographystyle{ieeetr}
\providecommand{\mbf}{\mathbf}
\providecommand{\pr}[1]{\ensuremath{\Pr\left(#1\right)}}
\providecommand{\qfunc}[1]{\ensuremath{Q\left(#1\right)}}
\providecommand{\sbrak}[1]{\ensuremath{{}\left[#1\right]}}
\providecommand{\lsbrak}[1]{\ensuremath{{}\left[#1\right.}}
\providecommand{\rsbrak}[1]{\ensuremath{{}\left.#1\right]}}
\providecommand{\brak}[1]{\ensuremath{\left(#1\right)}}
\providecommand{\lbrak}[1]{\ensuremath{\left(#1\right.}}
\providecommand{\rbrak}[1]{\ensuremath{\left.#1\right)}}
\providecommand{\cbrak}[1]{\ensuremath{\left\{#1\right\}}}
\providecommand{\lcbrak}[1]{\ensuremath{\left\{#1\right.}}
\providecommand{\rcbrak}[1]{\ensuremath{\left.#1\right\}}}

\providecommand{\system}{\overset{\mathcal{H}}{ \longleftrightarrow}}
	%\newcommand{\solution}[2]{\textbf{Solution:}{#1}}
\newcommand{\solution}{\noindent \textbf{Solution: }}
\newcommand{\cosec}{\,\text{cosec}\,}
\providecommand{\dec}[2]{\ensuremath{\overset{#1}{\underset{#2}{\gtrless}}}}
\newcommand{\myvec}[1]{\ensuremath{\begin{pmatrix}#1\end{pmatrix}}}
\newcommand{\mydet}[1]{\ensuremath{\begin{vmatrix}#1\end{vmatrix}}}
%\numberwithin{equation}{section}
\numberwithin{equation}{subsection}
%\numberwithin{problem}{section}
%\numberwithin{definition}{section}
\makeatletter
\@addtoreset{figure}{problem}
\makeatother
\let\StandardTheFigure\thefigure
\let\vec\mathbf
\usepackage{mathtools, nccmath}

\begin{document}

\begin{center}
\huge conjugate hyperbola\\

\large Shaik Zeeshan Ali\\
\large AI20MTECH11001\\
\end{center}
\begin{abstract}
This document is about matrix representation of lines and the bisectors of angles between them.
\end{abstract}
Download all python codes from 
\begin{lstlisting}
https://github.com/Zeeshan-IITH/IITH-EE5609/new/master/codes
\end{lstlisting}

and latex-tikz codes from 
\begin{lstlisting}
https://github.com/Zeeshan-IITH/IITH-EE5609
\end{lstlisting}
\section{problem}
Find the conjugate hyperbola to 
\begin{align}
    \vec{x}^T\vec{V}\vec{x}+2\vec{u}^T\vec{x}+f=0
\end{align}
where
\begin{align}
    \vec{V}=\myvec{a&b\\b&c}\\
    \vec{u}=\myvec{d\\e}
\end{align}
\section{construction}
The general quadratic equation of the form
\begin{align}
    ax^2+2bxy+cy^2+2dx+2ey+f=0
\end{align}
is a hyperbola when
\begin{align}
    \vec{V}<0\\
    \myvec{\vec{V}&\vec{u}\\\vec{u}^T&f}\neq 0
\end{align}
the center of the hyperbola will be $c$,where
\begin{align}
    c=-\vec{V}^{-1}\vec{u}
\end{align}
So,the matrix representation can be written as
\begin{align}
    \vec{x}^T\vec{V}\vec{x}-2c^T\vec{V}\vec{x}+f=0
\end{align}
\section{affine transformation}
The normalized eigen vector are used along with the center of the hyperbola to perform the affine transformation which brings the hyperbola into standard form.The affine transformation will be
\begin{align}
    \brak{\vec{P}\vec{y}+c}^T\vec{V}\brak{\vec{P}\vec{y}+c}-2c^T\vec{V}\brak{\vec{P}\vec{y}+c}+f=0
\end{align}
Simplifying it further we get
\begin{align}
    \vec{y}^T\vec{D}\vec{y}=c^T\vec{V}c-f\\
    \frac{1}{c^T\vec{V}c-f}\brak{\vec{y}^T\vec{D}\vec{y}}=1\label{eq:3.1}
\end{align}
where
\begin{align}
    \vec{D}=\vec{P}^T\vec{V}\vec{P}=\myvec{\lambda_1&0\\0&\lambda_2}
\end{align}
Here $\lambda_1,\lambda_2$ are the eigen values of $\vec{V}$.
\section{Explanation}
The Standard equation of hyperbola is 
\begin{align}
    \vec{y}^T\myvec{\frac{1}{a^2}&0\\0&-\frac{1}{b^2}}\vec{y}=1\\
    \frac{x}{a^2}-\frac{y}{b^2}=1
\end{align}
The conjugate hyperbola will be 
\begin{align}
    -\frac{x}{a^2}+\frac{y}{b^2}=1\\
    \vec{y}^T\myvec{-\frac{1}{a^2}&0\\0&\frac{1}{b^2}}\vec{y}=1\\
    \vec{y}^T\myvec{\frac{1}{a^2}&0\\0&-\frac{1}{b^2}}\vec{y}=-1
\end{align}
Therefore the conjugate hyperbola for the hyperbola in equation \eqref{eq:3.1} will be
\begin{align}
    \frac{1}{c^T\vec{V}c-f}\brak{\vec{y}^T\vec{D}\vec{y}}=-1\label{eq:4.1}\\
    \vec{y}^T\vec{D}\vec{y}=c^T\vec{V}c-f
\end{align}
Doing the inverse affine transformation we get,$\vec{y}=\vec{P}^{-1}\brak{\vec{x}-c}$
\begin{align}
    \brak{\vec{P}^{-1}\brak{\vec{x}-c}}^T\vec{D}\brak{\vec{P}^{-1}\brak{\vec{x}-c}}=f-c^T\vec{V}c\\
    \brak{\vec{x}-c}^T\vec{P}\vec{D}\vec{P}^T\brak{\vec{x}-c}=f-c^T\vec{V}c\\
    \brak{\vec{x}-c}^T\vec{V}\brak{\vec{x}-c}=f-c^T\vec{V}c\\
    \vec{x}^T\vec{V}\vec{x}-2c^T\vec{V}\vec{x}+c^T\vec{V}c=f-c^T\vec{V}c\\
    \vec{x}^T\vec{V}\vec{x}-2c^T\vec{V}\vec{x}+2c^T\vec{V}-f=0
\end{align}
Therefore the conjugate hyperbola of the $\vec{x}^T\vec{V}\vec{x}-2c^T\vec{V}\vec{x}+f=0$ is
\begin{align}
    \vec{x}^T\vec{V}\vec{x}-2c^T\vec{V}\vec{x}+2c^T\vec{V}c-f=0
\end{align}
\section{example}
A hyperbola of the $\vec{x}^T\vec{V}\vec{x}+2\vec{u}^T\vec{x}+f=0$ is
\begin{align}
    \vec{x}^T\myvec{8&5\\5&-3}\vec{x}+\myvec{-2&4}\vec{x}-2=0
\end{align}
The center of the hyperbola is at
\begin{align}
    c&=-\vec{V}^{-1}\vec{u}\\
    &=-\myvec{\frac{3}{49}&\frac{5}{49}\\\frac{5}{49}&-\frac{8}{49}}\myvec{-1\\2}\\
    &=\myvec{-\frac{1}{7}\\\frac{3}{7}}
\end{align}
The value of the constant will be
\begin{align}
    2c^T\vec{V}c-f&=2\myvec{-\frac{1}{7}&\frac{3}{7}}\myvec{8&5\\5&-3}\myvec{-\frac{1}{7}\\\frac{3}{7}}+2\\
    &=2\myvec{-\frac{1}{7}&\frac{3}{7}}\myvec{1\\-2}+2\\
    &=-2+2\\
    &=0
\end{align}
therefore the conjugate hyperbola is 
\begin{align}
    \vec{x}^T\myvec{8&5\\5&-3}\vec{x}+\myvec{-2&4}\vec{x}=0
\end{align}
\begin{figure}[h]
    \centering
    \includegraphics[width=\columnwidth]{Conjugate_hyperbola.png}
    \caption{Conjugate hyperbola}
    \label{fig:my_label}
\end{figure}
\end{document}

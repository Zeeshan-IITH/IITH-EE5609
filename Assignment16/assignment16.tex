  \documentclass[journal,12pt]{IEEEtran}
\usepackage{longtable}
\usepackage{setspace}
\usepackage{gensymb}
\singlespacing
\usepackage[cmex10]{amsmath}
\newcommand\myemptypage{
	\null
	\thispagestyle{empty}
	\addtocounter{page}{-1}
	\newpage
}
\usepackage{amsthm}
\usepackage{mdframed}
\usepackage{mathrsfs}
\usepackage{txfonts}
\usepackage{stfloats}
\usepackage{bm}
\usepackage{cite}
\usepackage{cases}
\usepackage{subfig}

\usepackage{longtable}
\usepackage{multirow}


\usepackage{enumitem}
\usepackage{mathtools}
\usepackage{steinmetz}
\usepackage{tikz}
\usepackage{circuitikz}
\usepackage{verbatim}
\usepackage{tfrupee}
\usepackage[breaklinks=true]{hyperref}
\usepackage{graphicx}
\usepackage{tkz-euclide}

\usetikzlibrary{calc,math}
\usepackage{listings}
    \usepackage{color}                                            %%
    \usepackage{array}                                            %%
    \usepackage{longtable}                                        %%
    \usepackage{calc}                                             %%
    \usepackage{multirow}                                         %%
    \usepackage{hhline}                                           %%
    \usepackage{ifthen}                                           %%
    \usepackage{lscape}     
\usepackage{multicol}
\usepackage{chngcntr}

\DeclareMathOperator*{\Res}{Res}

\renewcommand\thesection{\arabic{section}}
\renewcommand\thesubsection{\thesection.\arabic{subsection}}
\renewcommand\thesubsubsection{\thesubsection.\arabic{subsubsection}}

\renewcommand\thesectiondis{\arabic{section}}
\renewcommand\thesubsectiondis{\thesectiondis.\arabic{subsection}}
\renewcommand\thesubsubsectiondis{\thesubsectiondis.\arabic{subsubsection}}


\hyphenation{op-tical net-works semi-conduc-tor}
\def\inputGnumericTable{}                                 %%

\lstset{
%language=C,
frame=single, 
breaklines=true,
columns=fullflexible
}
\begin{document}
\onecolumn

\newtheorem{theorem}{Theorem}[section]
\newtheorem{problem}{Problem}
\newtheorem{proposition}{Proposition}[section]
\newtheorem{lemma}{Lemma}[section]
\newtheorem{corollary}[theorem]{Corollary}
\newtheorem{example}{Example}[section]
\newtheorem{definition}[problem]{Definition}

\newcommand{\BEQA}{\begin{eqnarray}}
\newcommand{\EEQA}{\end{eqnarray}}
\newcommand{\define}{\stackrel{\triangle}{=}}
\bibliographystyle{IEEEtran}
\raggedbottom
\setlength{\parindent}{0pt}
\providecommand{\mbf}{\mathbf}
\providecommand{\pr}[1]{\ensuremath{\Pr\left(#1\right)}}
\providecommand{\qfunc}[1]{\ensuremath{Q\left(#1\right)}}
\providecommand{\sbrak}[1]{\ensuremath{{}\left[#1\right]}}
\providecommand{\lsbrak}[1]{\ensuremath{{}\left[#1\right.}}
\providecommand{\rsbrak}[1]{\ensuremath{{}\left.#1\right]}}
\providecommand{\brak}[1]{\ensuremath{\left(#1\right)}}
\providecommand{\lbrak}[1]{\ensuremath{\left(#1\right.}}
\providecommand{\rbrak}[1]{\ensuremath{\left.#1\right)}}
\providecommand{\cbrak}[1]{\ensuremath{\left\{#1\right\}}}
\providecommand{\lcbrak}[1]{\ensuremath{\left\{#1\right.}}
\providecommand{\rcbrak}[1]{\ensuremath{\left.#1\right\}}}
\theoremstyle{remark}
\renewcommand{\arraystretch}{2}
\newtheorem{rem}{Remark}
\newcommand{\sgn}{\mathop{\mathrm{sgn}}}
\providecommand{\abs}[1]{\mathbf{\left\vert#1\right\vert}}
\providecommand{\res}[1]{\Res\displaylimits_{#1}} 
\providecommand{\norm}[1]{\mathbf{\left\lVert#1\right\rVert}}
%\providecommand{\norm}[1]{\lVert#1\rVert}
\providecommand{\mtx}[1]{\mathbf{#1}}
\providecommand{\mean}[1]{\mathbf{E\left[ #1 \right]}}
\providecommand{\fourier}{\overset{\mathcal{F}}{ \rightleftharpoons}}
%\providecommand{\hilbert}{\overset{\mathcal{H}}{ \rightleftharpoons}}
\providecommand{\system}{\overset{\mathcal{H}}{ \longleftrightarrow}}
	%\newcommand{\solution}[2]{\textbf{Solution:}{#1}}
\newcommand{\solution}{\noindent \textbf{Solution: }}
\newcommand{\cosec}{\,\text{cosec}\,}
\providecommand{\dec}[2]{\ensuremath{\overset{#1}{\underset{#2}{\gtrless}}}}
\newcommand{\myvec}[1]{\ensuremath{\begin{pmatrix}#1\end{pmatrix}}}
\newcommand{\mydet}[1]{\ensuremath{\begin{vmatrix}#1\end{vmatrix}}}
\numberwithin{equation}{subsection}
\makeatletter
\@addtoreset{figure}{problem}
\makeatother
\let\StandardTheFigure\thefigure
\let\vec\mathbf
\renewcommand{\thefigure}{\theproblem}
\def\putbox#1#2#3{\makebox[0in][l]{\makebox[#1][l]{}\raisebox{\baselineskip}[0in][0in]{\raisebox{#2}[0in][0in]{#3}}}}
     \def\rightbox#1{\makebox[0in][r]{#1}}
     \def\centbox#1{\makebox[0in]{#1}}
     \def\topbox#1{\raisebox{-\baselineskip}[0in][0in]{#1}}
     \def\midbox#1{\raisebox{-0.5\baselineskip}[0in][0in]{#1}}
\vspace{2cm}
\begin{center}
\huge Assignment 16\\
\large Shaik Zeeshan Ali\\
\large AI20MTECH11001\\
\end{center}
Download all python codes from 
\begin{lstlisting}
https://github.com/Zeeshan-IITH/IITH-EE5609/new/master/codes
\end{lstlisting}
and latex-tikz codes from 
\begin{lstlisting}
https://github.com/Zeeshan-IITH/IITH-EE5609
\end{lstlisting}
\section{problem}
Let $\vec{V}$ be a finite-dimensional inner product space, and let $\vec{W}$ be a subspace of $\vec{V}$. Then $\vec{V}=\vec{W}\oplus\vec{W}^\perp$, that is, each $\alpha$ in $\vec{V}$ is uniquely expressible in the form $\alpha=\beta+\gamma$, with $\beta\in\vec{W}$ and $\gamma\in\vec{W}^\perp$. Define a linear operator $\vec{U}$ by $\vec{U}\alpha=\beta-\gamma$.
\begin{enumerate}
    \item Prove that $\vec{U}$ is both self-adjoint and unitary.
    \item If $\vec{V}$ is $\mathbb{R}^3$ with standard inner product and $\vec{W}$ is the subspace spanned by $\brak{1,0,1}$, find the matrix of $\vec{U}$ in the standard ordered basis.
\end{enumerate}
\section{construction}
\renewcommand{\thetable}{1}
\begin{longtable}{|l|l|}
    \hline
    Given&$\vec{V}$ is a finite-dimensional inner product space, and $\vec{W}$ is a subspace of $\vec{V}$. Then\\ &$\vec{V}=\vec{W}\oplus\vec{W}^\perp$
    that is, each $\alpha$ in $\vec{V}$ is uniquely expressible in the form $\alpha=\beta+\gamma$,\\
    &with $\beta\in\vec{W}$ and $\gamma\in\vec{W}^\perp$.\\
    \hline
    Inner product&An inner product on a vector space $\vec{V}$ is a function which assigns to each ordered\\
    &pair of vecors $\alpha,\beta\in\vec{V}$ a scalar $\brak{\alpha\vert \beta}$ in $\mathbb{F}$ in such a way that for all $\alpha,\beta\in\vec{V}$\\
    & and all scalar $c\in\mathbb{F}$\\
    &\qquad \qquad $i)$ $\brak{\alpha+\beta\vert\gamma}=\brak{\alpha\vert\gamma}+\brak{\beta\vert\gamma}$\\
    &\qquad \qquad $ii)$ $\brak{c\alpha\vert\beta}=c\brak{\alpha\vert\beta}$\\
    &\qquad \qquad $iii)$ $\brak{\beta\vert\alpha}=\brak{\overline{\alpha\vert\beta}}$, where the bar denotes complex conjugation\\
    &\qquad \qquad $iv)$ $\brak{\alpha\vert\alpha}>0$ if $\alpha\neq 0$\\
    \hline
    Inner product&Inner product space is a real or complex vector space together with a specified\\
    space&inner product on that space. If $\vec{V}$ is an inner product space, then for any vectors \\
    &$\alpha,\beta\in\vec{V}$ and any scalar $c$\\
    &\qquad \qquad $i)$ $\norm{c\alpha}=\abs{c}$ $\norm{\alpha}$\\
    \hline
    &\qquad\qquad $ii)$ $\norm{\alpha}>0$ for $\alpha\neq 0$\\
    &\qquad\qquad$iii)$ $\abs{\brak{\alpha\vert\beta}}\leq \norm{\alpha}$ $\norm{\beta}$\\
    &\qquad\qquad$iv)$ $\norm{\alpha+\beta}\leq \norm{\alpha}+\norm{\beta}$\\
    \hline
    Orthogonal&Given that $\vec{W}$ is a subspace of $\vec{V}$. The orthogonal complement of $\vec{W}$ is the \\
    complement & set $\vec{W}^\perp$ of all vectors in $\vec{V}$ which are orthogonal to every vector in $\vec{W}$.Given that\\
    & $\beta\in\vec{W}$ and $\gamma\in\vec{W}^\perp$, then the inner product $\brak{\beta\vert\gamma}$ will be equal to $0$.\\
    \hline
    \caption{Construction}
    \label{tab:Cons}
\end{longtable}
\section{Proof}
\renewcommand{\thetable}{2}
\begin{longtable}{|l|l|}
    \hline
    Theorem&For any linear operator $\vec{U}$ on a finite dimensional inner product space $\vec{V}$, there exists a\\
    & unique linear operator $\vec{U}^*$ on $\vec{V}$ such that\\
    &\qquad\qquad\qquad $\brak{\vec{U}\alpha\vert\beta}=\brak{\alpha\vert\vec{U}^*\beta}$\\
    &for all $\alpha,\beta\in\vec{V}$. Then we say that $\vec{U}$ has an adjoint on $\vec{V}$, which is $\vec{U}^*$\\
    \hline
    self adjoint& Let $\alpha_1,\alpha_2$ be any two arbitrary vectors in $\vec{V}$ such that $\alpha_1=\beta_1+\gamma_1$ and $\alpha_2=\beta_2+\gamma_2$\\
    & where $\beta_1,\beta_2\in\vec{W}$ and $\gamma_1,\gamma_2\in\vec{W}^\perp$. The linear operator $\vec{U}$ on the inner product space is\\
    &\qquad$\brak{\vec{U}\alpha_1\vert\alpha_2}=\brak{\beta_1-\gamma_1\vert\beta_2+\gamma_2}=\brak{\beta_1\vert\beta_2}+\brak{\beta_1\vert\gamma_2}-\brak{\gamma_1\vert\beta_2}-\brak{\gamma_1\vert\gamma_2}$\\
    &\qquad since $\vec{W}^\perp$ is orthogonal complement of $\vec{W}$\\
    &\qquad$\brak{\vec{U}\alpha_1\vert\alpha_2}=\brak{\beta_1\vert\beta_2}-\brak{\gamma_1\vert\gamma_2}$\\
    &similarly\\
    &\qquad$\brak{\alpha_1\vert\vec{U}\alpha_2}=\brak{\beta_1+\gamma_1\vert\beta_2-\gamma_2}=\brak{\beta_1\vert\beta_2}-\brak{\beta_1\vert\gamma_2}+\brak{\gamma_1\vert\beta_2}-\brak{\gamma_1\vert\gamma_2}$\\
    &\qquad since $\vec{W}^\perp$ is orthogonal complement of $\vec{W}$\\
    &\qquad$\brak{\alpha_1\vert\vec{U}\alpha_2}=\brak{\beta_1\vert\beta_2}-\brak{\gamma_1\vert\gamma_2}$\\
    & Since $\brak{\vec{U}\alpha_1\vert\alpha_2}=\brak{\alpha_1\vert\vec{U}\alpha_2}$, the linear operator $\vec{U}$ is self adjoint.\\
    \hline
    Theorem&Let $\vec{U}$ be a linear operator on the inner product space $\vec{V}$. Then $\vec{U}$ is unitary if and only\\
    &if the adjoint $\vec{U}^*$ of $\vec{U}$ exists and $\vec{U}\vec{U}^*=\vec{U}^*\vec{U}=\vec{I}$\\
    \hline
    Unitary& $\brak{\vec{U}\vec{U}^*}\alpha=\brak{\vec{U}\vec{U}}\alpha=\vec{U}\brak{\vec{U}\alpha}=\vec{U}^*\brak{\vec{U}}\alpha=\brak{\vec{U}^*\vec{U}}\alpha$\\
    &We know that $\vec{U}\alpha=\beta
    -\gamma$. Calculating $\vec{U}\vec{U}\alpha$, we get\\
    &$\vec{U}\vec{U}\alpha=\vec{U}\brak{\vec{U}\alpha}=\vec{U}\brak{\beta-\gamma}=\vec{U}\beta-\vec{U}\gamma=\beta+\gamma=\vec{I}\alpha$\\
    & Therefore as $\brak{\vec{U}\vec{U}^*}\alpha=\brak{\vec{U}^*\vec{U}}\alpha=\vec{U}\vec{U}\alpha=\vec{I}\alpha$\\
    &\qquad\qquad $\brak{\vec{U}\vec{U}^*}=\brak{\vec{U}^*\vec{U}}=\vec{I}$\\
    &So the linear operator is unitary.\\
    \hline
    Part 2&Given $\vec{V}$ is $\mathbb{R}^3$ with standard inner product and $\vec{W}$ is the subspace spanned by $\brak{1,0,1}$.\\
    &Let the vector in $\vec{W}^\perp$, the orthogonal complement of $\vec{W}$ be $\vec{a}=\brak{\gamma_1, \gamma_2, \gamma_3}$. Since $\vec{a}$ is \\
    &orthogonal to the subspace $\vec{W}$, we get the inner product as\\
    &\qquad$\brak{\vec{a}\vert(1,0,1)}=\gamma_1+\gamma_3=0$\\
    &If $\gamma_2=1$ and $\gamma_1=\gamma_3=0$ or $\gamma_2=0$ and $\gamma_1=-\gamma_3=1$, both the vectors $\brak{0,1,0}$ and\\
    &$\brak{1,0,-1}$ form the linearly independent basis of $\vec{W}^\perp$.Therefore\\
    &\qquad$\vec{W}=span\{\brak{1,0,1}\}$\\
    &\qquad$\vec{W}^\perp=span\{\brak{0,1,0},\brak{1,0,-1}\}$\\
    &Any vector $\alpha\in\vec{V}$ can be uniquely expressed as $\alpha=\beta+\gamma$ where $\beta\in\vec{W}$ and $\gamma\in\vec{W}^\perp$\\
    &So $\beta=x\brak{1,0,1}$ and $\gamma=y\brak{0,1,0}+z\brak{1,0,-1}$.\\
    &The standard basis $e_1,e_2,e_3$ can be represented in terms of $\vec{W}$ and $\vec{W}^\perp$ as\\
    &$e_1=\brak{1,0,0}=\beta_1+\gamma_1=x_1\brak{1,0,1}+y_1\brak{0,1,0}+z_1\brak{1,0,-1}=\brak{x_1+z_1,y_1,x_1-z_1}$\\
    & Therefore $e_1=\frac{1}{2}\brak{1,0,1}+\frac{1}{2}\brak{1,0,-1}$. Similarly\\
    &$e_2=\brak{0,1,0}$\\
    &$e_3=\frac{1}{2}\brak{1,0,1}-\frac{1}{2}\brak{1,0,-1}$\\
    &Representing the matrix of liner operator $\vec{U}$ with respect to standard basis\\
    &$\vec{U}=\myvec{\vec{Ue_1}&\vec{Ue_2}&\vec{Ue_3}}$\\
    &$\vec{Ue_1}=\vec{U}\brak{\frac{1}{2}\brak{1,0,1}+\frac{1}{2}\brak{1,0,-1}}={\frac{1}{2}\brak{1,0,1}-\frac{1}{2}\brak{1,0,-1}}=\brak{0,0,1}$\\
    \hline
    &$\vec{Ue_2}=\vec{U}\brak{0,1,0}=\brak{0,-1,0}$\\
    &$\vec{Ue_3}=\vec{U}\brak{\frac{1}{2}\brak{1,0,1}-\frac{1}{2}\brak{1,0,-1}}=\frac{1}{2}\brak{1,0,1}+\frac{1}{2}\brak{1,0,-1}=\brak{1,0,0}$\\
    &Therefore the matrix of linear operator is \\
    &\qquad\qquad$\vec{U}=\myvec{0&0&1\\0&-1&0\\1&0&0}$\\
    &Verifying we get\\
    &$\brak{\vec{U}e_1\vert \brak{1,1,1}}=\brak{\brak{0,0,1}\vert\brak{1,1,1}}=1$\\
    &$\brak{e_1\vert \vec{U}\brak{1,1,1}}=\brak{\brak{1,0,0}\vert\brak{1,-1,1}}=1$\\
    &$\vec{U}^2=\myvec{1&0&0\\0&1&0\\0&0&1}=\vec{I}$.Therefore the linear operator is unitary\\
    \hline
    \caption{Proof}
    \label{tab:Cons}
\end{longtable}
\end{document}

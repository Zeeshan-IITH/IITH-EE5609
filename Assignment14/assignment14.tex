\documentclass[journal,12pt]{IEEEtran}
\usepackage{longtable}
\usepackage{setspace}
\usepackage{gensymb}
\singlespacing
\usepackage[cmex10]{amsmath}
\newcommand\myemptypage{
	\null
	\thispagestyle{empty}
	\addtocounter{page}{-1}
	\newpage
}
\usepackage{amsthm}
\usepackage{mdframed}
\usepackage{mathrsfs}
\usepackage{txfonts}
\usepackage{stfloats}
\usepackage{bm}
\usepackage{cite}
\usepackage{cases}
\usepackage{subfig}

\usepackage{longtable}
\usepackage{multirow}


\usepackage{enumitem}
\usepackage{mathtools}
\usepackage{steinmetz}
\usepackage{tikz}
\usepackage{circuitikz}
\usepackage{verbatim}
\usepackage{tfrupee}
\usepackage[breaklinks=true]{hyperref}
\usepackage{graphicx}
\usepackage{tkz-euclide}

\usetikzlibrary{calc,math}
\usepackage{listings}
    \usepackage{color}                                            %%
    \usepackage{array}                                            %%
    \usepackage{longtable}                                        %%
    \usepackage{calc}                                             %%
    \usepackage{multirow}                                         %%
    \usepackage{hhline}                                           %%
    \usepackage{ifthen}                                           %%
    \usepackage{lscape}     
\usepackage{multicol}
\usepackage{chngcntr}

\DeclareMathOperator*{\Res}{Res}

\renewcommand\thesection{\arabic{section}}
\renewcommand\thesubsection{\thesection.\arabic{subsection}}
\renewcommand\thesubsubsection{\thesubsection.\arabic{subsubsection}}

\renewcommand\thesectiondis{\arabic{section}}
\renewcommand\thesubsectiondis{\thesectiondis.\arabic{subsection}}
\renewcommand\thesubsubsectiondis{\thesubsectiondis.\arabic{subsubsection}}


\hyphenation{op-tical net-works semi-conduc-tor}
\def\inputGnumericTable{}                                 %%

\lstset{
%language=C,
frame=single, 
breaklines=true,
columns=fullflexible
}
\begin{document}
\onecolumn

\newtheorem{theorem}{Theorem}[section]
\newtheorem{problem}{Problem}
\newtheorem{proposition}{Proposition}[section]
\newtheorem{lemma}{Lemma}[section]
\newtheorem{corollary}[theorem]{Corollary}
\newtheorem{example}{Example}[section]
\newtheorem{definition}[problem]{Definition}

\newcommand{\BEQA}{\begin{eqnarray}}
\newcommand{\EEQA}{\end{eqnarray}}
\newcommand{\define}{\stackrel{\triangle}{=}}
\bibliographystyle{IEEEtran}
\raggedbottom
\setlength{\parindent}{0pt}
\providecommand{\mbf}{\mathbf}
\providecommand{\pr}[1]{\ensuremath{\Pr\left(#1\right)}}
\providecommand{\qfunc}[1]{\ensuremath{Q\left(#1\right)}}
\providecommand{\sbrak}[1]{\ensuremath{{}\left[#1\right]}}
\providecommand{\lsbrak}[1]{\ensuremath{{}\left[#1\right.}}
\providecommand{\rsbrak}[1]{\ensuremath{{}\left.#1\right]}}
\providecommand{\brak}[1]{\ensuremath{\left(#1\right)}}
\providecommand{\lbrak}[1]{\ensuremath{\left(#1\right.}}
\providecommand{\rbrak}[1]{\ensuremath{\left.#1\right)}}
\providecommand{\cbrak}[1]{\ensuremath{\left\{#1\right\}}}
\providecommand{\lcbrak}[1]{\ensuremath{\left\{#1\right.}}
\providecommand{\rcbrak}[1]{\ensuremath{\left.#1\right\}}}
\theoremstyle{remark}
\renewcommand{\arraystretch}{2}
\newtheorem{rem}{Remark}
\newcommand{\sgn}{\mathop{\mathrm{sgn}}}
\providecommand{\abs}[1]{\mathbf{\left\vert#1\right\vert}}
\providecommand{\res}[1]{\Res\displaylimits_{#1}} 
\providecommand{\norm}[1]{\mathbf{\left\lVert#1\right\rVert}}
%\providecommand{\norm}[1]{\lVert#1\rVert}
\providecommand{\mtx}[1]{\mathbf{#1}}
\providecommand{\mean}[1]{\mathbf{E\left[ #1 \right]}}
\providecommand{\fourier}{\overset{\mathcal{F}}{ \rightleftharpoons}}
%\providecommand{\hilbert}{\overset{\mathcal{H}}{ \rightleftharpoons}}
\providecommand{\system}{\overset{\mathcal{H}}{ \longleftrightarrow}}
	%\newcommand{\solution}[2]{\textbf{Solution:}{#1}}
\newcommand{\solution}{\noindent \textbf{Solution: }}
\newcommand{\cosec}{\,\text{cosec}\,}
\providecommand{\dec}[2]{\ensuremath{\overset{#1}{\underset{#2}{\gtrless}}}}
\newcommand{\myvec}[1]{\ensuremath{\begin{pmatrix}#1\end{pmatrix}}}
\newcommand{\mydet}[1]{\ensuremath{\begin{vmatrix}#1\end{vmatrix}}}
\numberwithin{equation}{subsection}
\makeatletter
\@addtoreset{figure}{problem}
\makeatother
\let\StandardTheFigure\thefigure
\let\vec\mathbf
\renewcommand{\thefigure}{\theproblem}
\def\putbox#1#2#3{\makebox[0in][l]{\makebox[#1][l]{}\raisebox{\baselineskip}[0in][0in]{\raisebox{#2}[0in][0in]{#3}}}}
     \def\rightbox#1{\makebox[0in][r]{#1}}
     \def\centbox#1{\makebox[0in]{#1}}
     \def\topbox#1{\raisebox{-\baselineskip}[0in][0in]{#1}}
     \def\midbox#1{\raisebox{-0.5\baselineskip}[0in][0in]{#1}}
\vspace{2cm}
\begin{center}
\huge Assignment 14\\
\large Shaik Zeeshan Ali\\
\large AI20MTECH11001\\
\end{center}
Download all python codes from 
\begin{lstlisting}
https://github.com/Zeeshan-IITH/IITH-EE5609/new/master/codes
\end{lstlisting}
and latex-tikz codes from 
\begin{lstlisting}
https://github.com/Zeeshan-IITH/IITH-EE5609
\end{lstlisting}
\section{problem}
Let $p_n\brak{x}=x^n$ for $x\in\mathbb{R}$ and let $\varrho=span\{p_0,p_1,p_2,...\}$. Then
\begin{enumerate}
    \item $\varrho$ is a vector space of all real valued continuous functions on $\mathbb{R}$.
    \item $\varrho$ is a subspace of all real valued continuous functions on $\mathbb{R}$.
    \item $\{p_0,p_1,p_2,...\}$ is a linearly independent set in the vector space of all real valued continuous functions on $\mathbb{R}$.
    \item Trigonometric functions belong to $\varrho$.
\end{enumerate}
\section{construction}
\renewcommand{\thetable}{1}
\begin{longtable}{|l|l|}
    \hline
    Given & $p_n\brak{x}=x^n$ for $x\in\mathbb{R}$ and $\varrho=span\{p_0,p_1,p_2,...\}$.\\
    \hline
    Vector& The set $S$ consisting of all real continuous functions on $\mathbb{R}$ forms a vector space.\\
    space&Let $f$ and $g$ be two real continuous functions from the set $S$.\\
    of real&Since the sum of two continuous function is a continuous function.\\
    continuous&$i)$ Addition is commutative $f+g=g+f$\\
    functions&$ii)$ Addition is associative$f+(g+h)=(f+g)+h$\\
    on $\mathbb{R}$&$iii)$There is unique $O$, zero function which maps every element to 0.\\
    &$iv)$Additive inverse.For each $f$ in $S$, $-f$ is a function in $S$.\\
    &$v)$Properties of scalar multiplication.For $c,c_1,c_2\in \mathbb{R}$,\\
    &\qquad $a)$ $1f=f$ where the constant function $1$ maps every element to $1$.\\
    &\qquad $b)$ $(c_1c_2)f=c_1(c_2f)$\\
    &\qquad $c)$ $c(f+g)=cf+cg$\\
    &\qquad $d)$ $c_1+c_2)f=c_1f+c_2f$\\
    \hline
    Option 1& $\varrho$ represents the vector space of polynomials. Polynomial functions are infintely \\
    & continuously differentiable.So any function that is continuous but not differentiable can \\
    & not be represented by polynomials.\\
    & Example the function $\abs{x}$ is continous but cannot be represented in \\
    &polynomial basis.Therefore option 1 is incorrect.\\
    \hline
    Option 2& $\varrho$ forms a subspace of all real valued continuous function on $\mathbb{R}$\\
    &Let $\alpha,\beta$ be two polynomial functions of order m and n, represented by the tuple of\\ &coefficients $(a_0,a_2,a_2..a_m)$ and $(b_0,b_1,b_2...b_n)$,then\\
    &$c\alpha+\beta$ is also a polynomial function whose coefficients are $(ca_0+b_0,ca_1+b_1,ca_2+b_2...)$\\
    &Therefore $\varrho$ is a subspace of all real valued continuous functions on $\mathbb{R}$.\\
    \hline
    Option 3&Consider the expression\\
    &$a_0p_0+a_1p_1+a_2p_2+...=O\implies a_0=a_1=a_2=...=0$\\
    &Hence $\{p_0,p_1,p_2,..\}$ are linearly independent set in the vector space of all real valued \\
    &continuous functions on $\mathbb{R}$.\\
    \hline
    Option 4&The period of trigonometric functions is finite where as the period of polynomials is \\
    &infinite. So, they cannot belong to the same class.\\
    \hline
    \caption{Answer}
    \label{tab:Ans}
\end{longtable}
\end{document}

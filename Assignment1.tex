\documentclass[journal,12pt,twocolumn]{IEEEtran}

\usepackage{setspace}
\usepackage{gensymb}
\usepackage{amsmath}
\singlespacing

\usepackage{listings}

\lstset{
frame=single, 
breaklines=true,
columns=fullflexible
}


\begin{document}
\begin{center}
\huge Assignment 1\\
\\*
\\*

\large Shaik Zeeshan Ali\\
\normalize AI20MTECH11001\\
\end{center}
\begin{abstract}
This document explains the properties of a unit vector and how to find out if two vectors are perpendicular, using an example of three mutually perpendicular unit vectors
\end{abstract}

\\Download all python codes from 
\begin{lstlisting}
https://github.com/Zeeshan-IITH/IITH-EE5609/new/master/codes
\end{lstlisting}
%
\\and latex-tikz codes from 


\begin{lstlisting}
https://github.com/Zeeshan-IITH/IITH-EE5609
\end{lstlisting}
%
\section{Problem}
show that each of the given three vectors is a unit vector 
\(\frac{1}{7}\) \begin{pmatrix}2 \\3 \\6\end{pmatrix},
\(\frac{1}{7}\) \begin{pmatrix}3 \\-6 \\2\end{pmatrix},
\(\frac{1}{7}\) \begin{pmatrix}6 \\2 \\-3\end{pmatrix}.
\\Also show that they are mutually perpendicular to each other.

\section{Explanation}
A unit vector is a vector of unit magnitude.\\
let \begin{align}
    a$=\(\frac{1}{7}\) \begin{pmatrix}2 \\3 \\6\end{pmatrix},b$=\(\frac{1}{7}\) \begin{pmatrix}3 \\-6 \\2\end{pmatrix},c$=\(\frac{1}{7}\) \begin{pmatrix}6 \\2 \\-3\end{pmatrix}\\
\end{align}
\norm{\vec{a}}=1\\

\end{document}

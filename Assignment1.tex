\documentclass[journal,12pt,twocolumn]{IEEEtran}
\usepackage{mathtools}
\usepackage{setspace}
\usepackage{gensymb}
\usepackage{amsmath}
\usepackage{commath}
\singlespacing

\usepackage{listings}

\lstset{
frame=single, 
breaklines=true,
columns=fullflexible
}


\begin{document}
\begin{center}
\huge Assignment 1\\

\large Shaik Zeeshan Ali\\
\large AI20MTECH11001\\
\end{center}
\vspace{0.5cm}
\begin{abstract}
This document explains the properties of a unit vector and how to find out if two vectors are perpendicular, using an example of three mutually perpendicular unit vectors
\end{abstract}
\vspace{0.5cm}
Download all python codes from 
\begin{lstlisting}
https://github.com/Zeeshan-IITH/IITH-EE5609/new/master/codes
\end{lstlisting}
%
and latex-tikz codes from 
\begin{lstlisting}
https://github.com/Zeeshan-IITH/IITH-EE5609
\end{lstlisting}
%
\vspace{0.5cm}
\section{Problem}
show that each of the given three vectors is a unit vector

\(\frac{1}{7}\) $\begin{pmatrix}2 \\3 \\6\end{pmatrix}$,
\(\frac{1}{7}\) $\begin{pmatrix}3 \\-6 \\2\end{pmatrix}$,
\(\frac{1}{7}\) $\begin{pmatrix}6 \\2 \\-3\end{pmatrix}$.

\section{Explanation}
A unit vector is a vector of unit magnitude.
\vspace{0.5cm} let $\vec{a}$=\(\frac{1}{7}\) $\begin{pmatrix}2 \\3 \\6\end{pmatrix}$,$\vec{b}$=\(\frac{1}{7}\) $\begin{pmatrix}3 \\-6 \\2\end{pmatrix}$,$\vec{c}$=\(\frac{1}{7}\) $\begin{pmatrix}6 \\2 \\-3\end{pmatrix}$\\
\norm{\vec{a}}=\(\frac{1}{7}\)$\sqrt{2^2+3^2+6^2}$=1\\
\norm{\vec{b}}=\(\frac{1}{7}\)$\sqrt{3^2+-6^2+2^2}$=1\\
\norm{\vec{c}}=\(\frac{1}{7}\)$\sqrt{6^2+2^2+-3^2}$=1\\
\section{Problem}
Also show that the three vectors are mutually perpendicular to each other
\section{Explanation}
\noindent
When two vectors are perpendicular to each other their dot product is zero.\par
The dot product of $\vec{a}$ and $\vec{b}$ is\par
$A^TB$=\(\frac{1}{7}\)\cdot\(\frac{1}{7}\)(2\cdot3+3\cdot-6+6\cdot2)$=0\par
The dot product of $\vec{b}$ and $\vec{c}$ is \par
$B^TC$=$\(\frac{1}{7}\)\cdot\(\frac{1}{7}\)(2\cdot3+3\cdot-6+6\cdot2)$=0\par
The dot product of $\vec{c}$ and $\vec{a}$ is \par
$C^TA$=$\(\frac{1}{7}\)\cdot\(\frac{1}{7}\)(6\cdot2+2\cdot3+-3\cdot6)$=0
\end{document}

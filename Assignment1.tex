\documentclass[journal,12pt,twocolumn]{IEEEtran}

\usepackage{setspace}
\usepackage{gensymb}

\singlespacing

\usepackage[cmex10]{amsmath}
\usepackage{amsthm}
\usepackage{mathrsfs}
\usepackage{txfonts}
\usepackage{stfloats}
\usepackage{bm}
\usepackage{cite}
\usepackage{cases}
\usepackage{subfig}
\usepackage{longtable}
\usepackage{multirow}
\usepackage{mathtools}
\usepackage{steinmetz}
\usepackage{tikz}
\usepackage{circuitikz}
\usepackage{verbatim}
\usepackage{tfrupee}
\usepackage[breaklinks=true]{hyperref}
\usepackage{tkz-euclide} % loads  TikZ and tkz-base
%\usetkzobj{all}
\usetikzlibrary{calc,math}
\usepackage{listings}
    \usepackage{color}                                            %%
    \usepackage{array}                                            %%
    \usepackage{longtable}                                        %%
    \usepackage{calc}                                             %%
    \usepackage{multirow}                                         %%
    \usepackage{hhline}                                           %%
    \usepackage{ifthen}                                           %%
  %optionally (for landscape tables embedded in another document): %%
    \usepackage{lscape}     
\usepackage{multicol}
\usepackage{chngcntr}
\DeclareMathOperator*{\Res}{Res}
\renewcommand\thesection{\arabic{section}}
\renewcommand\thesubsection{\thesection.\arabic{subsection}}
\renewcommand\thesubsubsection{\thesubsection.\arabic{subsubsection}}

\renewcommand\thesectiondis{\arabic{section}}
\renewcommand\thesubsectiondis{\thesectiondis.\arabic{subsection}}
\renewcommand\thesubsubsectiondis{\thesubsectiondis.\arabic{subsubsection}}

\newcommand{\bignorm}[1]{\Bigl \| #1 \Bigr \| #1}
\newcommand{\norm}[1]{\| #1 \|}
% correct bad hyphenation here
\hyphenation{op-tical net-works semi-conduc-tor}
\def\inputGnumericTable{}                                 %%

\lstset{
frame=single, 
breaklines=true,
columns=fullflexible
}


\begin{document}
\begin{center}
\huge Assignment 1\\

\large Shaik Zeeshan Ali\\
\large AI20MTECH11001\\
\end{center}
\vspace{0.5cm}
\begin{abstract}
This document explains the properties of a unit vector and how to find out if two vectors are perpendicular, using an example of three mutually perpendicular unit vectors
\end{abstract}
\vspace{0.5cm}
Download all python codes from 
\begin{lstlisting}
https://github.com/Zeeshan-IITH/IITH-EE5609/new/master/codes
\end{lstlisting}
%
and latex-tikz codes from 
\begin{lstlisting}
https://github.com/Zeeshan-IITH/IITH-EE5609
\end{lstlisting}
%
\vspace{0.5cm}
\section{Problem}
show that each of the given three vectors is a unit vector
\begin{align}
     \frac{1}{7} \begin{pmatrix}2 \\3 \\6\end{pmatrix}\\
    \frac{1}{7}\begin{pmatrix}3 \\-6 \\2\end{pmatrix}\\
    \frac{1}{7}\begin{pmatrix}6 \\2 \\-3\end{pmatrix}
\end{align}
\section{Explanation}
A unit vector is a vector of unit magnitude.
\vspace{0.5cm} let $\bm{A}$=\(\frac{1}{7}\) $\begin{pmatrix}2 \\3 \\6\end{pmatrix}$,$\bm{B}$=\(\frac{1}{7}\) $\begin{pmatrix}3 \\-6 \\2\end{pmatrix}$,$\bm{C}$=\(\frac{1}{7}\) $\begin{pmatrix}6 \\2 \\-3\end{pmatrix}$\\
\begin{align}
    \norm{\bm{A}}=\frac{1}{7}\sqrt{2^2+3^2+6^2}=1\\
    \norm{\bm{B}}=\frac{1}{7}\sqrt{3^2+-6^2+2^2}=1\\
    \norm{\bm{C}}=\frac{1}{7}\sqrt{6^2+2^2+-3^2}=1
\end{align}
\section{Problem}
Also show that the three vectors are mutually perpendicular to each other
\section{Explanation}
When two vectors are perpendicular to each other their dot product is zero.
\par
\begin{align}
\text{The dot product of $\bm{A}$ and $\bm{B}$ is}\notag\\
\bm{A}^T\bm{B}=\frac{1}{7}\cdot\frac{1}{7}(2\cdot3+3\cdot-6+6\cdot2)=0\\
\text{The dot product of $\bm{B}$ and $\bm{C}$ is} \notag\\
\bm{B}^T\bm{C}=\frac{1}{7}\cdot\frac{1}{7}(2\cdot3+3\cdot-6+6\cdot2)=0\\
\text{The dot product of $\bm{C}$ and $\bm{A}$ is} \notag\\
\bm{C}^T\bm{A}=\frac{1}{7}\cdot\frac{1}{7}(6\cdot2+2\cdot3+-3\cdot6)=0
\end{align}
Hence, the three unit vectors are mutually perpendicular to each other.
\end{document}

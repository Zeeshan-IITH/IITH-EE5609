\documentclass[journal,12pt,twocolumn]{IEEEtran}
%
\usepackage{setspace}
\usepackage{textcomp}
\usepackage{gensymb}
%\doublespacing
\singlespacing

\usepackage[cmex10]{amsmath}
\usepackage{amsthm}
%\usepackage{iithtlc}
\usepackage{mathrsfs}
\usepackage{txfonts}
\usepackage{stfloats}
\usepackage{bm}
\usepackage{cite}
\usepackage{cases}
\usepackage{subfig}
%\usepackage{xtab}
\usepackage{longtable}
\usepackage{multirow}
%\usepackage{algorithm}
%\usepackage{algpseudocode}
\usepackage{enumitem}
\usepackage{mathtools}
\usepackage{steinmetz}
\usepackage{tikz}
\usepackage{circuitikz}
\usepackage{verbatim}
\usepackage{tfrupee}
\usepackage[breaklinks=true]{hyperref}
%\usepackage{stmaryrd}
\usepackage{tkz-euclide} % loads  TikZ and tkz-base
%\usetkzobj{all}
\usetikzlibrary{calc,math}
\usepackage{listings}
    \usepackage{color}                                            %%
    \usepackage{array}                                            %%
    \usepackage{longtable}                                        %%
    \usepackage{calc}                                             %%
    \usepackage{multirow}                                         %%
    \usepackage{hhline}                                           %%
    \usepackage{ifthen}                                           %%
  %optionally (for landscape tables embedded in another document): %%
    \usepackage{lscape}     
\usepackage{multicol}
\usepackage{chngcntr}
%\usepackage{enumerate}

%\usepackage{wasysym}
%\newcounter{MYtempeqncnt}
\DeclareMathOperator*{\Res}{Res}
%\renewcommand{\baselinestretch}{2}
\renewcommand\thesection{\arabic{section}}
\renewcommand\thesubsection{\thesection.\arabic{subsection}}
\renewcommand\thesubsubsection{\thesubsection.\arabic{subsubsection}}

\renewcommand\thesectiondis{\arabic{section}}
\renewcommand\thesubsectiondis{\thesectiondis.\arabic{subsection}}
\renewcommand\thesubsubsectiondis{\thesubsectiondis.\arabic{subsubsection}}

% correct bad hyphenation here
\hyphenation{op-tical net-works semi-conduc-tor}
\def\inputGnumericTable{}                                 %%

\lstset{
%language=C,
frame=single, 
breaklines=true,
columns=fullflexible
}
\newenvironment{amatrix}[1]{%
  \left(\begin{array}{@{}*{#1}{c}|c@{}}
}{%
  \end{array}\right)
}
\DeclarePairedDelimiter\abs{\lvert}{\rvert}%
\DeclarePairedDelimiter\norm{\lVert}{\rVert}%

% Swap the definition of \abs* and \norm*, so that \abs
% and \norm resizes the size of the brackets, and the 
% starred version does not.
\makeatletter
\let\oldabs\abs
\def\abs{\@ifstar{\oldabs}{\oldabs*}}
%
\let\oldnorm\norm
\def\norm{\@ifstar{\oldnorm}{\oldnorm*}}
\makeatother

\newtheorem{theorem}{Theorem}[section]
\newtheorem{problem}{Problem}
\newtheorem{proposition}{Proposition}[section]
\newtheorem{lemma}{Lemma}[section]
\newtheorem{corollary}[theorem]{Corollary}
\newtheorem{example}{Example}[section]
\newtheorem{definition}[problem]{Definition}
%\newtheorem{thm}{Theorem}[section] 
%\newtheorem{defn}[thm]{Definition}
%\newtheorem{algorithm}{Algorithm}[section]
%\newtheorem{cor}{Corollary}
\newcommand{\BEQA}{\begin{eqnarray}}
\newcommand{\EEQA}{\end{eqnarray}}
\newcommand{\define}{\stackrel{\triangle}{=}}
\bibliographystyle{IEEEtran}
%\bibliographystyle{ieeetr}
\providecommand{\mbf}{\mathbf}
\providecommand{\pr}[1]{\ensuremath{\Pr\left(#1\right)}}
\providecommand{\qfunc}[1]{\ensuremath{Q\left(#1\right)}}
\providecommand{\sbrak}[1]{\ensuremath{{}\left[#1\right]}}
\providecommand{\lsbrak}[1]{\ensuremath{{}\left[#1\right.}}
\providecommand{\rsbrak}[1]{\ensuremath{{}\left.#1\right]}}
\providecommand{\brak}[1]{\ensuremath{\left(#1\right)}}
\providecommand{\lbrak}[1]{\ensuremath{\left(#1\right.}}
\providecommand{\rbrak}[1]{\ensuremath{\left.#1\right)}}
\providecommand{\cbrak}[1]{\ensuremath{\left\{#1\right\}}}
\providecommand{\lcbrak}[1]{\ensuremath{\left\{#1\right.}}
\providecommand{\rcbrak}[1]{\ensuremath{\left.#1\right\}}}
\providecommand{\system}{\overset{\mathcal{H}}{ \longleftrightarrow}}
	%\newcommand{\solution}[2]{\textbf{Solution:}{#1}}
\newcommand{\solution}{\noindent \textbf{Solution: }}
\newcommand{\cosec}{\,\text{cosec}\,}
\providecommand{\dec}[2]{\ensuremath{\overset{#1}{\underset{#2}{\gtrless}}}}
\newcommand{\myvec}[1]{\ensuremath{\begin{pmatrix}#1\end{pmatrix}}}
\newcommand{\mydet}[1]{\ensuremath{\begin{vmatrix}#1\end{vmatrix}}}
%\numberwithin{equation}{section}
\numberwithin{equation}{subsection}
%\numberwithin{problem}{section}
%\numberwithin{definition}{section}
\makeatletter
\@addtoreset{figure}{problem}
\makeatother
\let\StandardTheFigure\thefigure
\let\vec\mathbf
\usepackage{mathtools, nccmath}
\usepackage{longtable}
\usepackage{multirow}

\usetikzlibrary{calc,math}
\usepackage{listings}
\usepackage{color}                                            %%
\usepackage{array}                                            %%
\usepackage{longtable}                                        %%
\usepackage{calc}                                             %%
\usepackage{multirow}                                         %%
\usepackage{hhline}                                           %%
\usepackage{ifthen}                                           %%
%optionally (for landscape tables embedded in another document): %%
\usepackage{lscape}     
\usepackage{multicol}
\usepackage{chngcntr}


\renewcommand{\arraystretch}{1.5}
\usepackage{afterpage}
\newcommand\myemptypage{
	\null
	\thispagestyle{empty}
	\addtocounter{page}{-1}
	\newpage
}

\begin{document}

\begin{center}
\huge Assignment 1\\

\large Shaik Zeeshan Ali\\
\large AI20MTECH11001\\
\end{center}
\begin{abstract}
This document is about inverse of the given matrices.
\end{abstract}
Download all python codes from 
\begin{lstlisting}
https://github.com/Zeeshan-IITH/IITH-EE5609/new/master/codes
\end{lstlisting}

and latex-tikz codes from 
\begin{lstlisting}
https://github.com/Zeeshan-IITH/IITH-EE5609
\end{lstlisting}
\section{problem}
Let $\vec{V}$ be the vector space of $n\times n$ matrices over field $\vec{F}$. Let $\vec{A}$ be a fixed $n\times n$ matrix. Let $\vec{T}$ be the linear operator on $\vec{V}$ defined by
\begin{align}
    \vec{T\brak{B}}=\vec{AB}
\end{align}
Show that the minimal polynomial for $\vec{T}$ is the minimal polynomial for $\vec{A}$.
\section{construction}
\begin{table}[h]
    \centering
    \begin{tabular}{|c|c|}
    \hline
        Given &  $\vec{A}$ is a fixed matrix from the vector space $\vec{V}$ of $n\times n$ matrices. A linear operator \\
        &on the finite dimensional vector space $\vec{V}$, $\vec{T}$ is defined as $\vec{T\brak{B}}=\vec{AB}$.\\
    \hline
        Minimal polynomial & The minimal polynomial of a linear operator $\vec{T}$ is a monic polynomial which \\
        &annihilates $\vec{T}$.\\
    \hline
        Matrix representation & If we stack up the columns of the matrix $\vec{B}$,the linear operator $\vec{T}$ can be \\
        of $\vec{T}$ & represented in the equivalent form as \\
        & If $\vec{B}=\myvec{b_1&b_2&.&.&b_n}$,then the linear transformation of $\vec{B}$ will be\\
        &$\vec{T\brak{B}}=\myvec{\vec{A}b_1&\vec{A}b_2&.&.\vec{A}b_n}$\\
        & $\vec{T\brak{B}}=\myvec{\vec{T}\brak{b_1}\\\vec{T}\brak{b_2}\\.\\.\\\vec{T}\brak{b_n}}=\myvec{\vec{A}&&&&\\&\vec{A}&&\vec{O}&\\&&.&&\\&\vec{O}&&.&\\&&&&\vec{A}}\myvec{b_1\\b_2\\.\\.\\b_n}$\\
         & here each element represents the elements of the matrix $\vec{AB}$\\
    \hline
    \end{tabular}
\end{table}
\pagebreak
\myemptypage
\begin{table}[h]
    \centering
    \begin{tabular}{|c|c|}
    \hline
         & $\vec{T}=\myvec{\vec{A}&&&&\\&\vec{A}&&\vec{O}&\\&&.&&\\&\vec{O}&&.&\\&&&&\vec{A}}$\\
    \hline
        Properties of minimal & The roots of the characteristic polynomial,eigen values and the minimal  \\
        polynomial &polynomial are same,except for multiplicities.The roots of\\
        & the minimal polynomial of $\vec{A}$ are the roots of $\det\brak{\vec{A}-\lambda\vec{I}}$\\
    \hline
        The roots of minimal & The roots of the minimal polynomial of $\vec{T}$ are the roots\\
        polynomial of $\vec{T}$& of $\det \brak{\vec{T}-\lambda\vec{I}}$\\
        &$\det \brak{\vec{T}-\lambda\vec{I}}=\mydet{\brak{\vec{A}-\lambda\vec{I}}&&&&\\&\brak{\vec{A}-\lambda\vec{I}}&&\vec{O}&\\&&.&&\\&\vec{O}&&.&\\&&&&\brak{\vec{A}-\lambda\vec{I}}}$\\
        &=$\brak{\det \brak{\vec{A}-\lambda\vec{I}}}^{n}$\\
        & Therfore we can see that the eigen values of $\vec{A}$ are also the eigen values \\
        &of the linear operator $\vec{T}$\\
    \hline
        Minimal polynomial of $\vec{T}$ & The minimal polynomial of $\vec{A}$ divides the characteristic polynomial of \\
        &$\vec{A}$ and $\vec{T}$. Let the minimal polynomial of $\vec{A}$ is of degree $p\leq n$\\
        & $f\brak{x}=a_0+a_1x+a_2x^2...a_px^p$ such that $f\brak{\vec{A}}=0$\\
        &$f\brak{\vec{T}}=a_0\vec{I}+a_1\vec{T}+a_2\vec{T}^2+..+a_p\vec{T}^p$\\
        &$f\brak{\vec{T}}=\myvec{f\brak{\vec{A}}&&&&\\&f\brak{\vec{A}}&&\vec{O}&\\&&.&&\\&\vec{O}&&.&\\&&&&f\brak{\vec{A}}}=\vec{O}_{n^2\times n^2}$\\
        &Therefore the minimal polynomial for $\vec{T}$ is the minimal polynomial\\
        & for $\vec{A}$.\\
    \hline
    \end{tabular}
\end{table}
\end{document}

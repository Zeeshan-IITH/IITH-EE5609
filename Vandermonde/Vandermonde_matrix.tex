\documentclass[journal,12pt,twocolumn]{IEEEtran}
%
\usepackage{setspace}
\usepackage{gensymb}
%\doublespacing
\singlespacing

\usepackage[cmex10]{amsmath}
\usepackage{amsthm}
%\usepackage{iithtlc}
\usepackage{mathrsfs}
\usepackage{txfonts}
\usepackage{stfloats}
\usepackage{bm}
\usepackage{cite}
\usepackage{cases}
\usepackage{subfig}
%\usepackage{xtab}
\usepackage{longtable}
\usepackage{multirow}
%\usepackage{algorithm}
%\usepackage{algpseudocode}
\usepackage{enumitem}
\usepackage{mathtools}
\usepackage{steinmetz}
\usepackage{tikz}
\usepackage{circuitikz}
\usepackage{verbatim}
\usepackage{tfrupee}
\usepackage[breaklinks=true]{hyperref}
%\usepackage{stmaryrd}
\usepackage{tkz-euclide} % loads  TikZ and tkz-base
%\usetkzobj{all}
\usetikzlibrary{calc,math}
\usepackage{listings}
    \usepackage{color}                                            %%
    \usepackage{array}                                            %%
    \usepackage{longtable}                                        %%
    \usepackage{calc}                                             %%
    \usepackage{multirow}                                         %%
    \usepackage{hhline}                                           %%
    \usepackage{ifthen}                                           %%
  %optionally (for landscape tables embedded in another document): %%
    \usepackage{lscape}     
\usepackage{multicol}
\usepackage{chngcntr}
%\usepackage{enumerate}

%\usepackage{wasysym}
%\newcounter{MYtempeqncnt}
\DeclareMathOperator*{\Res}{Res}
%\renewcommand{\baselinestretch}{2}
\renewcommand\thesection{\arabic{section}}
\renewcommand\thesubsection{\thesection.\arabic{subsection}}
\renewcommand\thesubsubsection{\thesubsection.\arabic{subsubsection}}

\renewcommand\thesectiondis{\arabic{section}}
\renewcommand\thesubsectiondis{\thesectiondis.\arabic{subsection}}
\renewcommand\thesubsubsectiondis{\thesubsectiondis.\arabic{subsubsection}}

% correct bad hyphenation here
\hyphenation{op-tical net-works semi-conduc-tor}
\def\inputGnumericTable{}                                 %%

\lstset{
%language=C,
frame=single, 
breaklines=true,
columns=fullflexible
}
\newenvironment{amatrix}[1]{%
  \left(\begin{array}{@{}*{#1}{c}|c@{}}
}{%
  \end{array}\right)
}
\DeclarePairedDelimiter\abs{\lvert}{\rvert}%
\DeclarePairedDelimiter\norm{\lVert}{\rVert}%

% Swap the definition of \abs* and \norm*, so that \abs
% and \norm resizes the size of the brackets, and the 
% starred version does not.
\makeatletter
\let\oldabs\abs
\def\abs{\@ifstar{\oldabs}{\oldabs*}}
%
\let\oldnorm\norm
\def\norm{\@ifstar{\oldnorm}{\oldnorm*}}
\makeatother

\newtheorem{theorem}{Theorem}[section]
\newtheorem{problem}{Problem}
\newtheorem{proposition}{Proposition}[section]
\newtheorem{lemma}{Lemma}[section]
\newtheorem{corollary}[theorem]{Corollary}
\newtheorem{example}{Example}[section]
\newtheorem{definition}[problem]{Definition}
%\newtheorem{thm}{Theorem}[section] 
%\newtheorem{defn}[thm]{Definition}
%\newtheorem{algorithm}{Algorithm}[section]
%\newtheorem{cor}{Corollary}
\newcommand{\BEQA}{\begin{eqnarray}}
\newcommand{\EEQA}{\end{eqnarray}}
\newcommand{\define}{\stackrel{\triangle}{=}}
\bibliographystyle{IEEEtran}
%\bibliographystyle{ieeetr}
\providecommand{\mbf}{\mathbf}
\providecommand{\pr}[1]{\ensuremath{\Pr\left(#1\right)}}
\providecommand{\qfunc}[1]{\ensuremath{Q\left(#1\right)}}
\providecommand{\sbrak}[1]{\ensuremath{{}\left[#1\right]}}
\providecommand{\lsbrak}[1]{\ensuremath{{}\left[#1\right.}}
\providecommand{\rsbrak}[1]{\ensuremath{{}\left.#1\right]}}
\providecommand{\brak}[1]{\ensuremath{\left(#1\right)}}
\providecommand{\lbrak}[1]{\ensuremath{\left(#1\right.}}
\providecommand{\rbrak}[1]{\ensuremath{\left.#1\right)}}
\providecommand{\cbrak}[1]{\ensuremath{\left\{#1\right\}}}
\providecommand{\lcbrak}[1]{\ensuremath{\left\{#1\right.}}
\providecommand{\rcbrak}[1]{\ensuremath{\left.#1\right\}}}

\providecommand{\system}{\overset{\mathcal{H}}{ \longleftrightarrow}}
	%\newcommand{\solution}[2]{\textbf{Solution:}{#1}}
\newcommand{\solution}{\noindent \textbf{Solution: }}
\newcommand{\cosec}{\,\text{cosec}\,}
\providecommand{\dec}[2]{\ensuremath{\overset{#1}{\underset{#2}{\gtrless}}}}
\newcommand{\myvec}[1]{\ensuremath{\begin{pmatrix}#1\end{pmatrix}}}
\newcommand{\mydet}[1]{\ensuremath{\begin{vmatrix}#1\end{vmatrix}}}
%\numberwithin{equation}{section}
\numberwithin{equation}{subsection}
%\numberwithin{problem}{section}
%\numberwithin{definition}{section}
\makeatletter
\@addtoreset{figure}{problem}
\makeatother
\let\StandardTheFigure\thefigure
\let\vec\mathbf


\begin{document}

\begin{center}
\huge Vandermonde matrix\\

\large Shaik Zeeshan Ali\\
\large AI20MTECH11001\\
\end{center}
\begin{abstract}
This document calculates the determinant of a vandermonde matrix.
\end{abstract}
Download all python codes from 
\begin{lstlisting}
https://github.com/Zeeshan-IITH/IITH-EE5609/new/master/codes
\end{lstlisting}

and latex-tikz codes from 
\begin{lstlisting}
https://github.com/Zeeshan-IITH/IITH-EE5609
\end{lstlisting}
\section{Problem}
Derive an expression for the determinant of a Vandermonde matrix
\begin{align}
    \vec{V}=\myvec{1&\alpha_1&\alpha_1^2&\cdots&\alpha_1^{n-1}\\1&\alpha_2&\alpha_2^2&\cdots&\alpha_2^{n-1}\\\vdots&\vdots&\vdots&\vdots&\vdots\\1&\alpha_n&\alpha_n^2&\cdots&\alpha_n^{n-1}}
\end{align}
\section{construction}
The given matrix is of the order $n\times n$.The determinant is given as
\begin{align}
    \det{\vec{V}}=\det{\myvec{1&\alpha_1&\alpha_1^2&\cdots&\alpha_1^{n-1}\\1&\alpha_2&\alpha_2^2&\cdots&\alpha_2^{n-1}\\\vdots&\vdots&\vdots&\vdots&\vdots\\1&\alpha_n&\alpha_n^2&\cdots&\alpha_n^{n-1}}}
\end{align}
By doing Row operations we can reduce it to the form
\begin{align}
    \det{\vec{V}}={\mydet{1&\alpha_1&\alpha_1^2&\cdots&\alpha_1^{n-1}\\1&\alpha_2&\alpha_2^2&\cdots&\alpha_2^{n-1}\\\vdots&\vdots&\vdots&\vdots&\vdots\\1&\alpha_n&\alpha_n^2&\cdots&\alpha_n^{n-1}}}\xleftrightarrow[i\neq 1]{{R_i=R_i-R_1}}\\
    =\mydet{1&\alpha_1&\alpha_1^2&\cdots&\alpha_1^{n-1}\\0&\alpha_2-\alpha_1&\alpha_2^2-\alpha_1^2&\cdots&\alpha_2^{n-1}-\alpha_1^{n-1}\\\vdots&\vdots&\vdots&\vdots&\vdots\\0&\alpha_{n}-\alpha_1&\alpha_{n}^2-\alpha_1^2&\cdots&\alpha_{n}^{n-1}-\alpha_1^{n-1}}
\end{align}
By doing column operations we can reduce it to the form
\begin{align}
    \mydet{1&\alpha_1&\alpha_1^2&\cdots&\alpha_1^{n-1}\\0&\alpha_2-\alpha_1&\alpha_2^2-\alpha_1^2&\cdots&\alpha_2^{n-1}-\alpha_1^{n-1}\\\vdots&\vdots&\vdots&\vdots&\vdots\\0&\alpha_{n}-\alpha_1&\alpha_{n}^2-\alpha_1^2&\cdots&\alpha_{n}^{n-1}-\alpha_1^{n-1}}\xleftrightarrow[i>1]{{C_i=C_i-C_{i-1}}}\\
    =\mydet{1&0&0&\cdots&0\\0&\alpha_2-\alpha_1&(\alpha_2-\alpha_1)\alpha_2&\cdots&(\alpha_2-\alpha_1)\alpha_2^{n-2}\\\vdots&\vdots&\vdots&\vdots&\vdots\\0&\alpha_{n}-\alpha_1&(\alpha_{n}-\alpha_1)\alpha_{n}&\cdots&(\alpha_{n}-\alpha_1)\alpha_{n}^{n-2}}\\
    =\prod_{n\geq j >1} (\alpha_{j}-\alpha_{1}) {\mydet{1&\alpha_2&\alpha_2^2&\cdots&\alpha_2^{n-2}\\1&\alpha_3&\alpha_3^2&\cdots&\alpha_3^{n-2}\\\vdots&\vdots&\vdots&\vdots&\vdots\\1&\alpha_{n}&\alpha_{n}^2&\cdots&\alpha_{n}^{n-2}}}\label{eq:2}
\end{align}
\section{explanation}
By using equation \eqref{eq:2} the determinant can be reduced to $\brak{n-1}\times \brak{n-1}$, which is also of the form vandermonde matrix.The determinant of the reduced vandermonde matrix will be
\begin{align}
    {\mydet{1&\alpha_2&\alpha_2^2&\cdots&\alpha_2^{n-2}\\1&\alpha_3&\alpha_3^2&\cdots&\alpha_3^{n-2}\\\vdots&\vdots&\vdots&\vdots&\vdots\\1&\alpha_{n}&\alpha_{n}^2&\cdots&\alpha_{n}^{n-2}}}\\
    =\prod_{n\geq j >2} (\alpha_{j}-\alpha_{2}) {\mydet{1&\alpha_3&\alpha_3^2&\cdots&\alpha_3^{n-3}\\1&\alpha_4&\alpha_4^2&\cdots&\alpha_4^{n-3}\\\vdots&\vdots&\vdots&\vdots&\vdots\\1&\alpha_{n}&\alpha_{n}^2&\cdots&\alpha_{n}^{n-3}}}
\end{align}
So in each calculation the dimension is reduced.Continuing the similar reduction, the determinant of the matrix $\vec{V}$,we can write it as
\begin{align}
    &\det{\vec{V}}\\
    &=\prod_{1< j\leq n} (\alpha_{j}-\alpha_{1})\prod_{2< j\leq n}(\alpha_{j}-\alpha_{2})\cdots\prod_{(n-1)< j\leq n} (\alpha_{j}-\alpha_{n-1})\\
    &=\prod_{(1\leq i<n)}\prod_{(i< j\leq n)}(\alpha_{j}-\alpha_{i})
\end{align}
\end{document}

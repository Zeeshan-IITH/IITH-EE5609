\documentclass[journal,12pt,twocolumn]{IEEEtran}
%
\usepackage{setspace}
\usepackage{gensymb}
%\doublespacing
\singlespacing

\usepackage[cmex10]{amsmath}
\usepackage{amsthm}
%\usepackage{iithtlc}
\usepackage{mathrsfs}
\usepackage{txfonts}
\usepackage{stfloats}
\usepackage{bm}
\usepackage{cite}
\usepackage{cases}
\usepackage{subfig}
%\usepackage{xtab}
\usepackage{longtable}
\usepackage{multirow}
%\usepackage{algorithm}
%\usepackage{algpseudocode}
\usepackage{enumitem}
\usepackage{mathtools}
\usepackage{steinmetz}
\usepackage{tikz}
\usepackage{circuitikz}
\usepackage{verbatim}
\usepackage{tfrupee}
\usepackage[breaklinks=true]{hyperref}
%\usepackage{stmaryrd}
\usepackage{tkz-euclide} % loads  TikZ and tkz-base
%\usetkzobj{all}
\usetikzlibrary{calc,math}
\usepackage{listings}
    \usepackage{color}                                            %%
    \usepackage{array}                                            %%
    \usepackage{longtable}                                        %%
    \usepackage{calc}                                             %%
    \usepackage{multirow}                                         %%
    \usepackage{hhline}                                           %%
    \usepackage{ifthen}                                           %%
  %optionally (for landscape tables embedded in another document): %%
    \usepackage{lscape}     
\usepackage{multicol}
\usepackage{chngcntr}
%\usepackage{enumerate}

%\usepackage{wasysym}
%\newcounter{MYtempeqncnt}
\DeclareMathOperator*{\Res}{Res}
%\renewcommand{\baselinestretch}{2}
\renewcommand\thesection{\arabic{section}}
\renewcommand\thesubsection{\thesection.\arabic{subsection}}
\renewcommand\thesubsubsection{\thesubsection.\arabic{subsubsection}}

\renewcommand\thesectiondis{\arabic{section}}
\renewcommand\thesubsectiondis{\thesectiondis.\arabic{subsection}}
\renewcommand\thesubsubsectiondis{\thesubsectiondis.\arabic{subsubsection}}

% correct bad hyphenation here
\hyphenation{op-tical net-works semi-conduc-tor}
\def\inputGnumericTable{}                                 %%

\lstset{
%language=C,
frame=single, 
breaklines=true,
columns=fullflexible
}
\newenvironment{amatrix}[1]{%
  \left(\begin{array}{@{}*{#1}{c}|c@{}}
}{%
  \end{array}\right)
}
\DeclarePairedDelimiter\abs{\lvert}{\rvert}%
\DeclarePairedDelimiter\norm{\lVert}{\rVert}%

% Swap the definition of \abs* and \norm*, so that \abs
% and \norm resizes the size of the brackets, and the 
% starred version does not.
\makeatletter
\let\oldabs\abs
\def\abs{\@ifstar{\oldabs}{\oldabs*}}
%
\let\oldnorm\norm
\def\norm{\@ifstar{\oldnorm}{\oldnorm*}}
\makeatother

\newtheorem{theorem}{Theorem}[section]
\newtheorem{problem}{Problem}
\newtheorem{proposition}{Proposition}[section]
\newtheorem{lemma}{Lemma}[section]
\newtheorem{corollary}[theorem]{Corollary}
\newtheorem{example}{Example}[section]
\newtheorem{definition}[problem]{Definition}
%\newtheorem{thm}{Theorem}[section] 
%\newtheorem{defn}[thm]{Definition}
%\newtheorem{algorithm}{Algorithm}[section]
%\newtheorem{cor}{Corollary}
\newcommand{\BEQA}{\begin{eqnarray}}
\newcommand{\EEQA}{\end{eqnarray}}
\newcommand{\define}{\stackrel{\triangle}{=}}
\bibliographystyle{IEEEtran}
%\bibliographystyle{ieeetr}
\providecommand{\mbf}{\mathbf}
\providecommand{\pr}[1]{\ensuremath{\Pr\left(#1\right)}}
\providecommand{\qfunc}[1]{\ensuremath{Q\left(#1\right)}}
\providecommand{\sbrak}[1]{\ensuremath{{}\left[#1\right]}}
\providecommand{\lsbrak}[1]{\ensuremath{{}\left[#1\right.}}
\providecommand{\rsbrak}[1]{\ensuremath{{}\left.#1\right]}}
\providecommand{\brak}[1]{\ensuremath{\left(#1\right)}}
\providecommand{\lbrak}[1]{\ensuremath{\left(#1\right.}}
\providecommand{\rbrak}[1]{\ensuremath{\left.#1\right)}}
\providecommand{\cbrak}[1]{\ensuremath{\left\{#1\right\}}}
\providecommand{\lcbrak}[1]{\ensuremath{\left\{#1\right.}}
\providecommand{\rcbrak}[1]{\ensuremath{\left.#1\right\}}}

\providecommand{\system}{\overset{\mathcal{H}}{ \longleftrightarrow}}
	%\newcommand{\solution}[2]{\textbf{Solution:}{#1}}
\newcommand{\solution}{\noindent \textbf{Solution: }}
\newcommand{\cosec}{\,\text{cosec}\,}
\providecommand{\dec}[2]{\ensuremath{\overset{#1}{\underset{#2}{\gtrless}}}}
\newcommand{\myvec}[1]{\ensuremath{\begin{pmatrix}#1\end{pmatrix}}}
\newcommand{\mydet}[1]{\ensuremath{\begin{vmatrix}#1\end{vmatrix}}}
%\numberwithin{equation}{section}
\numberwithin{equation}{subsection}
%\numberwithin{problem}{section}
%\numberwithin{definition}{section}
\makeatletter
\@addtoreset{figure}{problem}
\makeatother
\let\StandardTheFigure\thefigure
\let\vec\mathbf


\begin{document}

\begin{center}
\huge Challenge 6\\

\large Shaik Zeeshan Ali\\
\large AI20MTECH11001\\
\end{center}
\begin{abstract}
This document is to prove that convolution is a unique map.
\end{abstract}
Download all python codes from 
\begin{lstlisting}
https://github.com/Zeeshan-IITH/IITH-EE5609/new/master/codes
\end{lstlisting}

and latex-tikz codes from 
\begin{lstlisting}
https://github.com/Zeeshan-IITH/IITH-EE5609
\end{lstlisting}
\section{problem}
Given two signals $\brak{x_0,…,x_n−1}$ and $\brak{h_0,…,h_m−1}$, the (linear) convolution of the two is an m+n−1-length signal  defined as
\begin{align}
    y\brak{t}=\brak{h\ast x}_t=\sum_{\tau=0}^{\tau=n-1} x_{\tau} h_\brak{t-\tau}\label{eq:1}\\
    0\leq t < m+n-1\notag
\end{align}
If 
\begin{align}
    y(n)=\brak{h_1\ast x}\label{eq:2}\\
    y(n)=\brak{h_2\ast x}\label{eq:3}
\end{align}
then prove that $h_1=h_2$
\section{construction}
Writing the convolution operation in matrix form 
\begin{align}
    Y=\myvec{h_0&0&0&.&.&0&0\\h_1&h_0&0&.&.&0&0\\h_2&h_1&h_0&.&.&0&0\\.&.&.&.&.&.&.\\h_{n-1}&h_{n-2}&n_{n-3}&.&.&h_1&h_0\\.&.&.&.&.&.&.\\h_{m-1}&h_{m-2}&h_{m-3}&.&.&h_{m-n+1}&h_{m-n}\\0&h_{m-1}&h_{m-2}&.&.&h_{m-n+2}&h_{m-n+1}\\0&0&h_{m-1}&.&.&h_{m-n+3}&h_{m-n+2}\\.&.&.&.&.&.&.\\0&0&0&.&.&0&h_{m-1}}\myvec{x_0\\x_1\\x_2.\\.\\.\\x_{n-1}}
\end{align}
\section{Explanation}
Therefore we can write equation \eqref{eq:1} in matrix form as $\vec{Y}=\vec{H}\vec{X}$ where
\begin{align}
    Y=\myvec{h_0&0&0&.&.&0&0\\h_1&h_0&0&.&.&0&0\\h_2&h_1&h_0&.&.&0&0\\.&.&.&.&.&.&.\\h_{n-1}&h_{n-2}&n_{n-3}&.&.&h_1&h_0\\.&.&.&.&.&.&.\\h_{m-1}&h_{m-2}&h_{m-3}&.&.&h_{m-n+1}&h_{m-n}\\0&h_{m-1}&h_{m-2}&.&.&h_{m-n+2}&h_{m-n+1}\\0&0&h_{m-1}&.&.&h_{m-n+3}&h_{m-n+2}\\.&.&.&.&.&.&.\\0&0&0&.&.&0&h_{m-1}}\myvec{x_0\\x_1\\x_2.\\.\\.\\x_{n-1}}
\end{align}
So the equations \eqref{eq:2} and \eqref{eq:3} can be written in matrix form where
\begin{align}
    \vec{H_1}=\myvec{h_{10}&0&0&.&.&0&0\\h_{11}&h_{10}&0&.&.&0&0\\h_{12}&h_{11}&h_{10}&.&.&0&0\\.&.&.&.&.&.&.\\h_{1n-1}&h_{1n-2}&n_{1n-3}&.&.&h_{11}&h_{10}\\.&.&.&.&.&.&.\\h_{1m-1}&h_{1m-2}&h_{1m-3}&.&.&h_{1m-n+1}&h_{1m-n}\\0&h_{1m-1}&h_{1m-2}&.&.&h_{1m-n+2}&h_{1m-n+1}\\0&0&h_{1m-1}&.&.&h_{1m-n+3}&h_{1m-n+2}\\.&.&.&.&.&.&.\\0&0&0&.&.&0&h_{1m-1}}\\
    \vec{H_1}=\myvec{h_{20}&0&0&.&.&0&0\\h_{21}&h_{20}&0&.&.&0&0\\h_{22}&h_{21}&h_{20}&.&.&0&0\\.&.&.&.&.&.&.\\h_{2n-1}&h_{2n-2}&n_{2n-3}&.&.&h_{21}&h_{20}\\.&.&.&.&.&.&.\\h_{2m-1}&h_{2m-2}&h_{2m-3}&.&.&h_{2m-n+1}&h_{2m-n}\\0&h_{2m-1}&h_{2m-2}&.&.&h_{2m-n+2}&h_{2m-n+1}\\0&0&h_{2m-1}&.&.&h_{2m-n+3}&h_{2m-n+2}\\.&.&.&.&.&.&.\\0&0&0&.&.&0&h_{2m-1}}
\end{align}
So
\begin{align}
    \brak{h_1\ast x}-\brak{h_2\ast x}=y(n)-y(n)=0\\
    (\vec{H_1}-\vec{H_2})\vec{X}=0
\end{align}
For the sake of simplicity lets assume that $m=n$,then Toeplitz matrix is of the form
\begin{align}
    \vec{H}=\myvec{L\\U}
\end{align}
Because $\vec{H_1},\vec{H_2}$ are in toeplitz form,their difference is also in toeplitz form.
\begin{align}
    \vec{H}=\vec{H_1}-\vec{H_2}\\
    \vec{H}=\myvec{h_0&0&0&.&.&0&0\\h_1&h_0&0&.&.&0&0\\h_2&h_1&h_0&.&.&0&0\\.&.&.&.&.&.&.\\h_{m-2}&h_{m-3}&.&.&h_{1}&h_{0}&0\\h_{m-1}&h_{m-2}&h_{m-3}&.&.&h_{1}&h_{0}\\0&h_{m-1}&h_{m-2}&.&.&h_{2}&h_{1}\\0&0&h_{m-1}&.&.&h_{3}&h_{2}\\.&.&.&.&.&.&.\\0&0&0&.&.&0&h_{m-1}}
\end{align}
Suppose \begin{align}
    L=\myvec{h_0&0&0&.&.&0&0\\h_1&h_0&0&.&.&0&0\\h_2&h_1&h_0&.&.&0&0\\.&.&.&.&.&.&.\\h_{n-2}&h_{n-3}&.&.&h_{1}&h_{0}&0\\h_{n-1}&h_{n-2}&h_{n-3}&.&.&h_{1}&h_{0}}\label{eq:4}\\
    LX=0
\end{align}
For $LX=0$ to have a non-trivial solution i.e nullity$\neq 0$, the rank of the matrix $R(L) < n$, which implies $h_0=0$.
Similarly if you consider the submatrix of $L$ 
\begin{align}
    L=\myvec{h_1&0&0&.&.&0&0\\h_2&h_1&0&.&.&0&0\\h_3&h_2&h_1&.&.&0&0\\.&.&.&.&.&.&.\\h_{n-2}&h_{n-3}&.&.&h_{2}&h_{2}&0\\h_{n-1}&h_{n-2}&h_{n-3}&.&.&h_{2}&h_{1}}\label{eq:5}
    \intertext{where}
    \myvec{h_1&0&0&.&.&0&0\\h_2&h_1&0&.&.&0&0\\h_3&h_2&h_1&.&.&0&0\\.&.&.&.&.&.&.\\h_{n-2}&h_{n-3}&.&.&h_{2}&h_{2}&0\\h_{n-1}&h_{n-2}&h_{n-3}&.&.&h_{2}&h_{1}}\myvec{x_1\\x_2.\\.\\.\\x_{n-1}}=0
\end{align}
By using the similar logic,$h_1=0$. And all the elements of the lower triangular matrix are zero i.e $L=0$,rank(L)=0 which implies $H=0$.\par
The solution for $LX=0$ or $UX=0$ when $X\neq 0$, is that each of the elements of the matrices $L,U$ is zero irrespective of $X$ because $X$, the input signal does not depend on the system.So,$H=0$, $\vec{H_1}-\vec{H_2}=0$ which means $\vec{H_1}=\vec{H_2}$
\end{document}

\documentclass[journal,12pt,twocolumn]{IEEEtran}
%
\usepackage{setspace}
\usepackage{gensymb}
%\doublespacing
\singlespacing

\usepackage[cmex10]{amsmath}
\usepackage{amsthm}
%\usepackage{iithtlc}
\usepackage{mathrsfs}
\usepackage{txfonts}
\usepackage{stfloats}
\usepackage{bm}
\usepackage{cite}
\usepackage{cases}
\usepackage{subfig}
%\usepackage{xtab}
\usepackage{longtable}
\usepackage{multirow}
%\usepackage{algorithm}
%\usepackage{algpseudocode}
\usepackage{enumitem}
\usepackage{mathtools}
\usepackage{steinmetz}
\usepackage{tikz}
\usepackage{circuitikz}
\usepackage{verbatim}
\usepackage{tfrupee}
\usepackage[breaklinks=true]{hyperref}
%\usepackage{stmaryrd}
\usepackage{tkz-euclide} % loads  TikZ and tkz-base
%\usetkzobj{all}
\usetikzlibrary{calc,math}
\usepackage{listings}
    \usepackage{color}                                            %%
    \usepackage{array}                                            %%
    \usepackage{longtable}                                        %%
    \usepackage{calc}                                             %%
    \usepackage{multirow}                                         %%
    \usepackage{hhline}                                           %%
    \usepackage{ifthen}                                           %%
  %optionally (for landscape tables embedded in another document): %%
    \usepackage{lscape}     
\usepackage{multicol}
\usepackage{chngcntr}
%\usepackage{enumerate}

%\usepackage{wasysym}
%\newcounter{MYtempeqncnt}
\DeclareMathOperator*{\Res}{Res}
%\renewcommand{\baselinestretch}{2}
\renewcommand\thesection{\arabic{section}}
\renewcommand\thesubsection{\thesection.\arabic{subsection}}
\renewcommand\thesubsubsection{\thesubsection.\arabic{subsubsection}}

\renewcommand\thesectiondis{\arabic{section}}
\renewcommand\thesubsectiondis{\thesectiondis.\arabic{subsection}}
\renewcommand\thesubsubsectiondis{\thesubsectiondis.\arabic{subsubsection}}

% correct bad hyphenation here
\hyphenation{op-tical net-works semi-conduc-tor}
\def\inputGnumericTable{}                                 %%

\lstset{
%language=C,
frame=single, 
breaklines=true,
columns=fullflexible
}
\newenvironment{amatrix}[1]{%
  \left(\begin{array}{@{}*{#1}{c}|c@{}}
}{%
  \end{array}\right)
}
\DeclarePairedDelimiter\abs{\lvert}{\rvert}%
\DeclarePairedDelimiter\norm{\lVert}{\rVert}%

% Swap the definition of \abs* and \norm*, so that \abs
% and \norm resizes the size of the brackets, and the 
% starred version does not.
\makeatletter
\let\oldabs\abs
\def\abs{\@ifstar{\oldabs}{\oldabs*}}
%
\let\oldnorm\norm
\def\norm{\@ifstar{\oldnorm}{\oldnorm*}}
\makeatother

\newtheorem{theorem}{Theorem}[section]
\newtheorem{problem}{Problem}
\newtheorem{proposition}{Proposition}[section]
\newtheorem{lemma}{Lemma}[section]
\newtheorem{corollary}[theorem]{Corollary}
\newtheorem{example}{Example}[section]
\newtheorem{definition}[problem]{Definition}
%\newtheorem{thm}{Theorem}[section] 
%\newtheorem{defn}[thm]{Definition}
%\newtheorem{algorithm}{Algorithm}[section]
%\newtheorem{cor}{Corollary}
\newcommand{\BEQA}{\begin{eqnarray}}
\newcommand{\EEQA}{\end{eqnarray}}
\newcommand{\define}{\stackrel{\triangle}{=}}
\bibliographystyle{IEEEtran}
%\bibliographystyle{ieeetr}
\providecommand{\mbf}{\mathbf}
\providecommand{\pr}[1]{\ensuremath{\Pr\left(#1\right)}}
\providecommand{\qfunc}[1]{\ensuremath{Q\left(#1\right)}}
\providecommand{\sbrak}[1]{\ensuremath{{}\left[#1\right]}}
\providecommand{\lsbrak}[1]{\ensuremath{{}\left[#1\right.}}
\providecommand{\rsbrak}[1]{\ensuremath{{}\left.#1\right]}}
\providecommand{\brak}[1]{\ensuremath{\left(#1\right)}}
\providecommand{\lbrak}[1]{\ensuremath{\left(#1\right.}}
\providecommand{\rbrak}[1]{\ensuremath{\left.#1\right)}}
\providecommand{\cbrak}[1]{\ensuremath{\left\{#1\right\}}}
\providecommand{\lcbrak}[1]{\ensuremath{\left\{#1\right.}}
\providecommand{\rcbrak}[1]{\ensuremath{\left.#1\right\}}}

\providecommand{\system}{\overset{\mathcal{H}}{ \longleftrightarrow}}
	%\newcommand{\solution}[2]{\textbf{Solution:}{#1}}
\newcommand{\solution}{\noindent \textbf{Solution: }}
\newcommand{\cosec}{\,\text{cosec}\,}
\providecommand{\dec}[2]{\ensuremath{\overset{#1}{\underset{#2}{\gtrless}}}}
\newcommand{\myvec}[1]{\ensuremath{\begin{pmatrix}#1\end{pmatrix}}}
\newcommand{\mydet}[1]{\ensuremath{\begin{vmatrix}#1\end{vmatrix}}}
%\numberwithin{equation}{section}
\numberwithin{equation}{subsection}
%\numberwithin{problem}{section}
%\numberwithin{definition}{section}
\makeatletter
\@addtoreset{figure}{problem}
\makeatother
\let\StandardTheFigure\thefigure
\let\vec\mathbf


\begin{document}

\begin{center}
\huge Challenge 5\\

\large Shaik Zeeshan Ali\\
\large AI20MTECH11001\\
\end{center}
\begin{abstract}
This document is to prove that the sides opposite to equal angles of a triangle are equal.
\end{abstract}
Download all python codes from 
\begin{lstlisting}
https://github.com/Zeeshan-IITH/IITH-EE5609/new/master/codes
\end{lstlisting}

and latex-tikz codes from 
\begin{lstlisting}
https://github.com/Zeeshan-IITH/IITH-EE5609
\end{lstlisting}
\section{Problem}
Prove that sides opposite to equal angles of a triangle are equal. 
\section{construction}
Consider the triangle in $XY$ plane.So the points $A,B,C$ are the three points of the triangle which has two angles equal.Let $\angle ABC$=$\angle ACB$.\par
Let $\theta$ be the angle made by $\angle ABC$ and $\angle ACB$ .The matrix which rotates the vector by an angle $\theta$ is
\begin{align}
    \vec{R_{\theta}}=\myvec{\cos{\theta} & -\sin{\theta}\\\sin{\theta} &\cos{\theta}}\label{eq:1}
\end{align}
Let $\vec{m_B{_C}}$ be the direction vector parallel to the line $BC$.
So the equation of the line parallel to $\vec{AB}$ and $\vec{AC}$ will be
\begin{align}
    L_1 \colon x=\vec{B}+\lambda_1\vec{R_{\theta}}\vec{m_B{_C}}\label{eq:2}\\
    L_2 \colon x=\vec{C}-\lambda_2\vec{R_{-\theta}}\vec{m_B{_C}}\label{eq:3}
\end{align}
The equation of a line that is perpendicular to the line $BC$ and passing through the midpoint of $B$ and $C$ is
\begin{align}
    L_3 \colon x=\frac{\vec{B+C}}{2}+\lambda_3\vec{R_{90}}\vec{m_B{_C}}\label{eq:4}\\
    \intertext{Where}
    \vec{R_{90}}=\myvec{0&-1\\1&0&}\notag
\end{align}
If there exists a point that satisfies all the three equations then the sides opposite to equal angles will be equal because the perpendicular from that point will pass through the midpoint of $B$ and $C$.Let $A$ be a point which satisfies all the three equations \eqref{eq:2},\eqref{eq:3} and \eqref{eq:4}.
\section{Explanation}
The point $A$ satisfies the equations \eqref{eq:2}, \eqref{eq:3} and \eqref{eq:4} we get
\begin{align}
    \vec{A}=\vec{B}+\lambda_1\vec{R_{\theta}}\vec{m_B{_C}}\label{eq:5}\\
    \vec{A}=\vec{C}-\lambda_2\vec{R_{-\theta}}\vec{m_B{_C}}\label{eq:6}\\
    \vec{A}=\frac{\vec{B+C}}{2}+\lambda_3\vec{R_{90}}\vec{m_B{_C}}\label{eq:7}
\end{align}
Writing the above equations in matrix form we get
\begin{align}
    \myvec{\vec{R_{\theta}}\vec{m_B{_C}}&0&0\\0&\vec{R_{-\theta}}\vec{m_B{_C}}&0\\0&0&\vec{R_{90}}\vec{m_B{_C}}}\myvec{\lambda_1\\\lambda_2\\\lambda_3}=\myvec{\vec{A-B}\\\vec{C-A}\\\vec{A-\frac{B+C}{2}}}
    \intertext{Where}
    \vec{R}=\myvec{\vec{R_{\theta}}\vec{m_B{_C}}&0&0\\0&\vec{R_{-\theta}}\vec{m_B{_C}}&0\\0&0&\vec{R_{90}}\vec{m_B{_C}}}\label{eq:8}
\end{align}
If there exist a unique set of values for $\lambda_1,\lambda_2,\lambda_3$ for a given points $A,B,C$,Then the matrix $R$ is non-singular.
Let
\begin{align}
    \vec{m_B{_C}}=\myvec{m_1\\m_2}\\
    \vec{R_\theta}\vec{m_B{_C}}=\myvec{m_1\cos{\theta}-m_2\sin{\theta}\\m_1\sin{\theta}+m_2\cos{\theta}}\label{eq:9}
\end{align}
The equation \eqref{eq:9}=0 only when $m_1=0$ and $m_2=0$.So the matrix $\vec{R}$ is always non-singular when $\theta\neq0$ and $\vec{m_B{_C}}\neq0$.
Let the point $M$ be the midpoint of the line $\vec{BC}$.Then the two triangles $\triangle AMB$ and $\triangle AMC$ both form a right angle at the point $M$,with the $\angle ABM=\angle ACM=\theta$.\par
Hence the sides opposite to equal angles of a triangle are equal.
\end{document}

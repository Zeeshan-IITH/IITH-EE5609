\documentclass[journal,12pt,twocolumn]{IEEEtran}

\usepackage{setspace}
\usepackage{gensymb}

\singlespacing

\usepackage[cmex10]{amsmath}
\usepackage{amsthm}
\usepackage{mathrsfs}
\usepackage{txfonts}
\usepackage{stfloats}
\usepackage{bm}
\usepackage{cite}
\usepackage{cases}
\usepackage{subfig}
\usepackage{longtable}
\usepackage{multirow}
\usepackage{mathtools}
\usepackage{steinmetz}
\usepackage{tikz}
\usepackage{circuitikz}
\usepackage{verbatim}
\usepackage{tfrupee}
\usepackage[breaklinks=true]{hyperref}
\usepackage{tkz-euclide} % loads  TikZ and tkz-base
%\usetkzobj{all}
\usetikzlibrary{calc,math}
\usepackage{listings}
    \usepackage{color}                                            %%
    \usepackage{array}                                            %%
    \usepackage{longtable}                                        %%
    \usepackage{calc}                                             %%
    \usepackage{multirow}                                         %%
    \usepackage{hhline}                                           %%
    \usepackage{ifthen}                                           %%
  %optionally (for landscape tables embedded in another document): %%
    \usepackage{lscape}    
\usepackage{graphicx}
\usepackage{multicol}
\usepackage{chngcntr}
\DeclareMathOperator*{\Res}{Res}
\renewcommand\thesection{\arabic{section}}
\renewcommand\thesubsection{\thesection.\arabic{subsection}}
\renewcommand\thesubsubsection{\thesubsection.\arabic{subsubsection}}

\renewcommand\thesectiondis{\arabic{section}}
\renewcommand\thesubsectiondis{\thesectiondis.\arabic{subsection}}
\renewcommand\thesubsubsectiondis{\thesubsectiondis.\arabic{subsubsection}}

\newcommand{\bignorm}[1]{\Bigl \| #1 \Bigr \| #1}
\newcommand{\norm}[1]{\| #1 \|}
% correct bad hyphenation here
\hyphenation{op-tical net-works semi-conduc-tor}
\def\inputGnumericTable{}                                 %%

\lstset{
frame=single, 
breaklines=true,
columns=fullflexible
}
\newcommand{\myvect}[1]{\ensuremath{\begin{pmatrix}#1\end{pmatrix}}}
\begin{document}
    \begin{center}
    \huge Challenge 2\\
    \large Shaik Zeeshan Ali\\
    \large AI20MTECH11001\\
    \end{center}
\vspace{0.5cm}
\begin{abstract}
This document is to know the conditions where matrix multiplication can be commutative
\end{abstract}
\vspace{0.5cm}
Download all python codes from 
\begin{lstlisting}
https://github.com/Zeeshan-IITH/IITH-EE5609/new/master/codes
\end{lstlisting}
%
and latex-tikz codes from 
\begin{lstlisting}
https://github.com/Zeeshan-IITH/IITH-EE5609
\end{lstlisting}
%
\section{problem}
The conditions when matrix multiplication can be commutative,especially where both the matrices are simultaneously diagonalizable.
\section{explanation}
Let $\bm{P}$ be an invertible matrix that can simultaneously diagonalize matrices $\bm{A}$ and $\bm{B}$.\\
Let $\bm{P}$=\myvect{\bm{X_1} & \bm{X_2} ...& \bm{X_n}} where $\bm{X_1}$,$\bm{X_2}$ ..$\bm{X_n}$ are the eigen vectors.\\
Then \\
\begin{align}
    \bm{A}\bm{X_1}=\lambda_a{_1}\bm{X_1},\bm{B}\bm{X_1}=\lambda_b{_1}\bm{X_1}\notag\\
    \bm{A}\bm{X_2}=\lambda_a{_2}\bm{X_2},\bm{B}\bm{X_2}=\lambda_b{_2}\bm{X_2}\notag \intertext{ and so on}\notag\\
    \bm{A}\bm{X_n}=\lambda_a{_n}\bm{X_n},\bm{B}\bm{X_1}=\lambda_b{_n}\bm{X_n}
\end{align}
But using the above equations we can write\\
\begin{align}
    \bm{A}(\bm{B}^{-1}\lambda_b{_1}\bm{X_1})=\lambda_a{_1}\bm{X_1}\notag\\
    \bm{A}(\bm{B}^{-1}\lambda_b{_2}\bm{X_2})=\lambda_a{_2}\bm{X_2}\notag\\
    \intertext{and so on}
    \bm{A}(\bm{B}^{-1}\lambda_b{_n}\bm{X_n})=\lambda_a{_n}\bm{X_n}
\end{align}
Since $\bm{X_1},\bm{X_2}..\bm{X_n}$ are not zero matrices\\
We can write\\
\begin{align}
    \lambda_b{_1}\bm{A}=\lambda_a{_1}\bm{B}\notag\\
    \lambda_b{_2}\bm{A}=\lambda_a{_2}\bm{B}\notag\\
    \intertext{and so on}\notag\\
    \lambda_b{_n}\bm{A}=\lambda_a{_n}\bm{B}
\end{align}
Therefore this is possible only if \\
\begin{align}
    \frac{\lambda_a{_1}}{\lambda_b{_1}}=\frac{\lambda_a{_2}}{\lambda_b{_2}}=....=\frac{\lambda_a{_n}}{\lambda_b{_n}}
    \intertext{and}\notag\\
    \intertext{Matrix $\bm{A}$ is a scalar multiple of $\bm{B}$}\notag
\end{align}
\end{document}

\documentclass[journal,12pt,twocolumn]{IEEEtran}

\usepackage{setspace}
\usepackage{gensymb}

\singlespacing

\usepackage[cmex10]{amsmath}
\usepackage{amsthm}
\usepackage{mathrsfs}
\usepackage{txfonts}
\usepackage{stfloats}
\usepackage{bm}
\usepackage{cite}
\usepackage{cases}
\usepackage{subfig}
\usepackage{longtable}
\usepackage{multirow}
\usepackage{mathtools}
\usepackage{steinmetz}
\usepackage{tikz}
\usepackage{circuitikz}
\usepackage{verbatim}
\usepackage{tfrupee}
\usepackage[breaklinks=true]{hyperref}
\usepackage{tkz-euclide} % loads  TikZ and tkz-base
%\usetkzobj{all}
\usetikzlibrary{calc,math}
\usepackage{listings}
    \usepackage{color}                                            %%
    \usepackage{array}                                            %%
    \usepackage{longtable}                                        %%
    \usepackage{calc}                                             %%
    \usepackage{multirow}                                         %%
    \usepackage{hhline}                                           %%
    \usepackage{ifthen}                                           %%
  %optionally (for landscape tables embedded in another document): %%
    \usepackage{lscape}     
\usepackage{multicol}
\usepackage{chngcntr}
\DeclareMathOperator*{\Res}{Res}
\renewcommand\thesection{\arabic{section}}
\renewcommand\thesubsection{\thesection.\arabic{subsection}}
\renewcommand\thesubsubsection{\thesubsection.\arabic{subsubsection}}

\renewcommand\thesectiondis{\arabic{section}}
\renewcommand\thesubsectiondis{\thesectiondis.\arabic{subsection}}
\renewcommand\thesubsubsectiondis{\thesubsectiondis.\arabic{subsubsection}}

\newcommand{\bignorm}[1]{\Bigl \| #1 \Bigr \| #1}
\newcommand{\norm}[1]{\| #1 \|}
% correct bad hyphenation here
\hyphenation{op-tical net-works semi-conduc-tor}
\def\inputGnumericTable{}                                 %%

\lstset{
frame=single, 
breaklines=true,
columns=fullflexible
}


\begin{document}
\begin{center}
\huge Assignment 2\\

\large Shaik Zeeshan Ali\\
\large AI20MTECH11001\\
\end{center}
\vspace{0.5cm}
\begin{abstract}
This document explains how to find the shortest distance between two lines if and when the two lines are not intersecting with each other.
\end{abstract}
\vspace{0.5cm}
Download all python codes from 
\begin{lstlisting}
https://github.com/Zeeshan-IITH/IITH-EE5609/new/master/codes
\end{lstlisting}
%
and latex-tikz codes from 
\begin{lstlisting}
https://github.com/Zeeshan-IITH/IITH-EE5609
\end{lstlisting}
%
\vspace{0.5cm}
\section{Problem}
Find the shortest distance between the lines \\
\begin{align}
    L_1\colon \bm{x}= \begin{pmatrix}1\\2\\1\end{pmatrix}+\lambda_1\begin{pmatrix}1\\-1\\1\end{pmatrix}\\
    L_2\colon \bm{x}= \begin{pmatrix}2\\-1\\-1\end{pmatrix}+\lambda_2\begin{pmatrix}2\\1\\2\end{pmatrix}
\end{align}
\section{construction}
When two lines are not intersecting the distance between them is non-zero.The equation of above mentioned lines in symmetric form is
\begin{align}
    L_1\colon x-1=2-y=z-1\\
    L_2\colon \frac{x-2}{2}=y+1=\frac{z+1}{2}
\end{align}
The above line equations have no point of intersection as for no value of $x,y,z$ both the equations (3) and (4) are satisfied.
\section{solution}
Let $\bm{A}$ be a point on line $L_1$ and $\bm{B}$ be point on the line $L_2$.Then the shortest distance between two skew lines will be the length of line perpendicular to both the lines $L_1$,$L_2$ and passing through $\bm{A}$ and $\bm{B}$.\par
The shortest distance between the lines will be the projection of any line between the points on $L_1$,$L_2$ on to the unit vector which is perpendicular to both $L_1$,$L_2$.\par
The unit vector perpendicular to lines \\
\begin{align}
    Line_1\colon \bm{x}=x_1+\lambda_1\bm{b_1}\notag\\
    Line_2\colon \bm{x}=x_2+\lambda_1\bm{b_2}\notag
\end{align}
can be found by calculating
\begin{align}
    \frac{\bm{b_1}\times\bm{b_2}}{\norm{\bm{b_1}\times\bm{b_2}}}\notag
\end{align}
In our question the value of $\bm{b_1}=\begin{pmatrix}1\\-1\\1\end{pmatrix}$ and $\bm{b_2}=\begin{pmatrix}2\\1\\2\end{pmatrix}$
So the unit vector perpendicular to both $L_1$ and $L_2$ is
\begin{align}
    \bm{u}=\frac{\begin{pmatrix}1\\-1\\1\end{pmatrix}\times\begin{pmatrix}2\\1\\2\end{pmatrix}}{\norm{\begin{pmatrix}1\\-1\\1\end{pmatrix}\times\begin{pmatrix}2\\1\\2\end{pmatrix}}}=\frac{1}{\sqrt{2}}\begin{pmatrix}-1\\0\\1\end{pmatrix}\notag
\end{align}
The points $\bm{A}=\begin{pmatrix}1\\2\\1\end{pmatrix}$ and $\bm{B}=\begin{pmatrix}2\\-1\\-1\end{pmatrix}$ lie on the line $L_1$,$L_2$ respectively.\par
The shortest distance between the lines is the absolute value of projection of the vector $\bm{B}-\bm{A}$ on to the unit vector $\bm{u}$.
\begin{equation}
    \norm{(\bm{B}-\bm{A})^T\bm{u}}\\
    =\norm{\frac{1}{\sqrt{2}}\begin{pmatrix}1\\-3\\-2\end{pmatrix}^T\begin{pmatrix}-1\\0\\1\end{pmatrix}}\\
    =\frac{3}{\sqrt{2}}\notag
\end{equation}
Therefore the shortest distance between the given lines is $\frac{3}{\sqrt{2}}$
\end{document}

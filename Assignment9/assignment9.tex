\documentclass[journal,12pt,twocolumn]{IEEEtran}
%
\usepackage{setspace}
\usepackage{textcomp}
\usepackage{gensymb}
%\doublespacing
\singlespacing

\usepackage[cmex10]{amsmath}
\usepackage{amsthm}
%\usepackage{iithtlc}
\usepackage{mathrsfs}
\usepackage{txfonts}
\usepackage{stfloats}
\usepackage{bm}
\usepackage{cite}
\usepackage{cases}
\usepackage{subfig}
%\usepackage{xtab}
\usepackage{longtable}
\usepackage{multirow}
%\usepackage{algorithm}
%\usepackage{algpseudocode}
\usepackage{enumitem}
\usepackage{mathtools}
\usepackage{steinmetz}
\usepackage{tikz}
\usepackage{circuitikz}
\usepackage{verbatim}
\usepackage{tfrupee}
\usepackage[breaklinks=true]{hyperref}
%\usepackage{stmaryrd}
\usepackage{tkz-euclide} % loads  TikZ and tkz-base
%\usetkzobj{all}
\usetikzlibrary{calc,math}
\usepackage{listings}
    \usepackage{color}                                            %%
    \usepackage{array}                                            %%
    \usepackage{longtable}                                        %%
    \usepackage{calc}                                             %%
    \usepackage{multirow}                                         %%
    \usepackage{hhline}                                           %%
    \usepackage{ifthen}                                           %%
  %optionally (for landscape tables embedded in another document): %%
    \usepackage{lscape}     
\usepackage{multicol}
\usepackage{chngcntr}
%\usepackage{enumerate}

%\usepackage{wasysym}
%\newcounter{MYtempeqncnt}
\DeclareMathOperator*{\Res}{Res}
%\renewcommand{\baselinestretch}{2}
\renewcommand\thesection{\arabic{section}}
\renewcommand\thesubsection{\thesection.\arabic{subsection}}
\renewcommand\thesubsubsection{\thesubsection.\arabic{subsubsection}}

\renewcommand\thesectiondis{\arabic{section}}
\renewcommand\thesubsectiondis{\thesectiondis.\arabic{subsection}}
\renewcommand\thesubsubsectiondis{\thesubsectiondis.\arabic{subsubsection}}

% correct bad hyphenation here
\hyphenation{op-tical net-works semi-conduc-tor}
\def\inputGnumericTable{}                                 %%

\lstset{
%language=C,
frame=single, 
breaklines=true,
columns=fullflexible
}
\newenvironment{amatrix}[1]{%
  \left(\begin{array}{@{}*{#1}{c}|c@{}}
}{%
  \end{array}\right)
}
\DeclarePairedDelimiter\abs{\lvert}{\rvert}%
\DeclarePairedDelimiter\norm{\lVert}{\rVert}%

% Swap the definition of \abs* and \norm*, so that \abs
% and \norm resizes the size of the brackets, and the 
% starred version does not.
\makeatletter
\let\oldabs\abs
\def\abs{\@ifstar{\oldabs}{\oldabs*}}
%
\let\oldnorm\norm
\def\norm{\@ifstar{\oldnorm}{\oldnorm*}}
\makeatother

\newtheorem{theorem}{Theorem}[section]
\newtheorem{problem}{Problem}
\newtheorem{proposition}{Proposition}[section]
\newtheorem{lemma}{Lemma}[section]
\newtheorem{corollary}[theorem]{Corollary}
\newtheorem{example}{Example}[section]
\newtheorem{definition}[problem]{Definition}
%\newtheorem{thm}{Theorem}[section] 
%\newtheorem{defn}[thm]{Definition}
%\newtheorem{algorithm}{Algorithm}[section]
%\newtheorem{cor}{Corollary}
\newcommand{\BEQA}{\begin{eqnarray}}
\newcommand{\EEQA}{\end{eqnarray}}
\newcommand{\define}{\stackrel{\triangle}{=}}
\bibliographystyle{IEEEtran}
%\bibliographystyle{ieeetr}
\providecommand{\mbf}{\mathbf}
\providecommand{\pr}[1]{\ensuremath{\Pr\left(#1\right)}}
\providecommand{\qfunc}[1]{\ensuremath{Q\left(#1\right)}}
\providecommand{\sbrak}[1]{\ensuremath{{}\left[#1\right]}}
\providecommand{\lsbrak}[1]{\ensuremath{{}\left[#1\right.}}
\providecommand{\rsbrak}[1]{\ensuremath{{}\left.#1\right]}}
\providecommand{\brak}[1]{\ensuremath{\left(#1\right)}}
\providecommand{\lbrak}[1]{\ensuremath{\left(#1\right.}}
\providecommand{\rbrak}[1]{\ensuremath{\left.#1\right)}}
\providecommand{\cbrak}[1]{\ensuremath{\left\{#1\right\}}}
\providecommand{\lcbrak}[1]{\ensuremath{\left\{#1\right.}}
\providecommand{\rcbrak}[1]{\ensuremath{\left.#1\right\}}}

\providecommand{\system}{\overset{\mathcal{H}}{ \longleftrightarrow}}
	%\newcommand{\solution}[2]{\textbf{Solution:}{#1}}
\newcommand{\solution}{\noindent \textbf{Solution: }}
\newcommand{\cosec}{\,\text{cosec}\,}
\providecommand{\dec}[2]{\ensuremath{\overset{#1}{\underset{#2}{\gtrless}}}}
\newcommand{\myvec}[1]{\ensuremath{\begin{pmatrix}#1\end{pmatrix}}}
\newcommand{\mydet}[1]{\ensuremath{\begin{vmatrix}#1\end{vmatrix}}}
%\numberwithin{equation}{section}
\numberwithin{equation}{subsection}
%\numberwithin{problem}{section}
%\numberwithin{definition}{section}
\makeatletter
\@addtoreset{figure}{problem}
\makeatother
\let\StandardTheFigure\thefigure
\let\vec\mathbf
\usepackage{mathtools, nccmath}
\usepackage{longtable}
\renewcommand{\arraystretch}{1.5}
\usepackage{afterpage}
\newcommand\myemptypage{
	\null
	\thispagestyle{empty}
	\addtocounter{page}{-1}
	\newpage
}

\begin{document}

\begin{center}
\huge Assignment 9\\

\large Shaik Zeeshan Ali\\
\large AI20MTECH11001\\
\end{center}
\begin{abstract}
This document is about positive definite properties of real symmetric non-singular matrix.
\end{abstract}
Download all python codes from 
\begin{lstlisting}
https://github.com/Zeeshan-IITH/IITH-EE5609/new/master/codes
\end{lstlisting}

and latex-tikz codes from 
\begin{lstlisting}
https://github.com/Zeeshan-IITH/IITH-EE5609
\end{lstlisting}
\section{problem}
For every $4\times 4$ real symmetric non-singular matrix $\vec{A}$,prove if there exists a positive integer $p$ such that
\begin{enumerate}
    \item $p\vec{I}+\vec{A}$ is positive definite
    \item $\vec{A}^p$ is positive definite
    \item $\vec{A}^{-p}$ is positive definite
    \item $exp(p\vec{A})-\vec{I}$ is positive definite
\end{enumerate}
\section{Construction}
\begin{table}[hp]
    \centering
    \begin{tabular}{|c|c|}
        \hline
        Definition  & An $n \times n$ symmetric real matrix $\vec{M}$ is said to be positive definite if $\vec{x}^T\vec{M}\vec{x}>0$ for all\\
        &non-zero $\vec{x}$ in $\mathbb{R}^n$ \\
        \hline
        Properties & An $n \times n$ symmetric real matrix always have real eigen vectors and the set of eigen \\& vectors can be selected such that they form an orthonormal basis\\
        & $\vec{M}=\vec{Q}\vec{\Lambda}\vec{Q}^T$ and $\vec{Q}\vec{Q}^T=\vec{I}$\\
        \hline
        Implication & An $n \times n$ symmetric real matrix $\vec{M}$ is said to be positive definite if\\
        & \qquad i) \quad Eigen values of $\vec{M}$ are all positive\\
        & \qquad ii) \quad The pivot elements of $\vec{M}$ are all positive\\
        \hline
    \end{tabular}
    \caption{Properties of positive definite matrix}
    \label{tab:my_label}
\end{table}
\pagebreak
\myemptypage
\begin{table}[h]
    \begin{tabular}{|c|c|}
        \hline
        Proof for 1 & \qquad $p\vec{I}+\vec{A}=\vec{Q}p\vec{I}\vec{Q}^T+\vec{Q}\vec{\Lambda}\vec{Q}^T=\vec{Q}\brak{p\vec{I}+\vec{\Lambda}}\vec{Q}^T$\\
        & if the eigen values of $\vec{A}$ are $\lambda_1,\lambda_2...\lambda_4$, then the eigen values of $p\vec{I}+\vec{A}$ will be\\ & $\lambda_1+p,\lambda_2+p...\lambda_4+p$.If we choose $p$ to be $> \lvert min \brak{\lambda_1,\lambda_2...\lambda_4}\rvert$ then $p\vec{I}+\vec{A}$ will be \\&positive definite.\\
        \hline
        Proof for 2 & $\vec{A}^p=\brak{\vec{Q}\vec{\Lambda}\vec{Q}^T}^p=\vec{Q}\vec{\Lambda}^p\vec{Q}^T$\\
        & if the eigen values of $\vec{A}$ are $\lambda_1,\lambda_2...\lambda_4$, then the eigen values of $\vec{A}^p$ will be\\& $\lambda_1^p,\lambda_2^p...\lambda_4^p$. If p is even then the eigen values are positive.\\
        \hline
        Proof for 3 & $\vec{A}^{-1}=\brak{\vec{Q}\vec{\Lambda}\vec{Q}^T}^{-1}=\vec{Q}\vec{\Lambda}^{-1}\vec{Q}^T$\\
        & if the eigen values of $\vec{A}$ are $\lambda_1,\lambda_2...\lambda_4\neq 0\brak{\text{For inverse to exist}}$, then the eigen values \\&of $\vec{A}^{-p}$ will be $\frac{1}{\lambda_1^p},\frac{1}{\lambda_2^p}...\frac{1}{\lambda_4^p}$. If p is even then the eigen values are positive.\\
        \hline
        Proof for 4 & $exp\brak{p\vec{A}}-\vec{I}=\sum_{k=0}^{\infty}\frac{1}{k!}\brak{p\vec{A}}^k-\vec{I}=p\vec{A}+\frac{1}{2!}\brak{p\vec{A}}^2+\frac{1}{3!}\brak{p\vec{A}}^3+...$\\
        &$=\vec{Q}\brak{p\vec{\Lambda}+\frac{1}{2!}\brak{p\vec{\Lambda}}^2+\frac{1}{3!}\brak{p\vec{\Lambda}}^3+...}\vec{Q}^T$\\
        &if the eigen values of $\vec{A}$ are $\lambda_1,\lambda_2...\lambda_4$, then the eigen values of \\
        &$exp\brak{p\vec{A}}-\vec{I}$ will be $e^{p\lambda_1}-1,e^{p\lambda_2}-1,..e^{p\lambda_4}-1$.So if the eigen value of $\vec{A}$\\
        &is negative then the corresponding eigen value of $exp\brak{p\vec{A}}-\vec{I}$ is also negative\\
        & for every positive integer $p$.So such a positive integer $p$ does not exist.\\
        \hline
    \end{tabular}
    \caption{PROOF}
    \label{tab:my_label}
\end{table}
\end{document}
